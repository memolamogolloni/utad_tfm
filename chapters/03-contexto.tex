\chapter{Contexto}
    Seddi es una compañia de computación gráfica especializada en simulación de tejidos para la
    industria de la moda.
    Su objetivo es crear una servicio en la nube en el que las empresa dedicadas a la moda puedan
    digitalizar todo su flujo de trabajo para agilizar el prototipado y reducir costes tanto económicos
    como medioambientales. Para ello se ofrece una plataforma online de trabajo colaborativo con
    herramientas que permiten editar el diseño o las propiedades de una prenda a los diferentes
    perfiles implicados en las fases de producción: diseñadores textiles, diseñadores de moda,
    patronistas, product managers, etc.

\section{Caracteristicas: especial atencion a los tejidos.}
    El motor de render online de Seddi debe ser capaz de mostrar un amplio abanico de materiales,
    pero prestando una especial atenci\'on a los tejidos. Los materiales que pueden ser usados durante
    el proceso de dise\~no de una prenda pueden ir desde cuero, lana, toalla, terciopelo... cada uno
    con unas propiedades opticas y mec\'anicas diferentes y \'unicas que influyen completamente en las
    decisiones de dise\~no del producto.

\section{Author: aplicacion web donde el usuario disenha sus creaciones}
    El proceso empieza con la creación de un digital twin de un hilo, capturando las propiedades
    físicas y mecánicas de las fibras que lo componen y que sirven como entradas a los algoritmos de
    simulación y render. A partir de este hilo los usuarios pueden diseñar un tejido o escoger entre
    una gran cantidad de ellos, en función de la frecuencia o el tipo de entrelazado de los hilos.
    Utilizando estos tejidos se diseñan patrones en plano con las formas de las piezas que conforman
    una prenda y se permite al usuario previsualizar y editar la construcción del modelo en una vista
    3D sobre un maniquí. Una vez completado el ciclo de diseño se generan automáticamente fichas técnicas
    con especificaciones sobre los elementos estéticos o constructivos que facilitan la comunicación
    entre todas las partes implicadas en el proceso de fabricación.
    
\section{Seddi pipeline: motor de render online y offline}
    \subsection{Offline}
    \subsection{Online}
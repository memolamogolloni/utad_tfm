\chapter{Contexto}
    Seddi es una compañia de computación gr\'afica especializada en simulaci\'on de tejidos para la
    industria de la moda.
    Su objetivo es crear una servicio en la nube en el que las empresa dedicadas a la moda puedan
    digitalizar todo su flujo de trabajo para agilizar el prototipado y reducir costes tanto econ\'omicos
    como medioambientales. Para ello se ofrece una plataforma online de trabajo colaborativo con
    herramientas que permiten editar el dise\~no o las propiedades de una prenda a los diferentes
    perfiles implicados en las fases de producci\'on: diseñadores textiles, diseñadores de moda,
    patronistas, product managers, etc.

\section{Caracteristicas: especial atenci\'on a los tejidos.}
    El motor de render online de Seddi debe ser capaz de mostrar un amplio abanico de materiales,
    pero prestando una especial atenci\'on a los tejidos. Los materiales que pueden ser usados durante
    el proceso de dise\~no de una prenda pueden ir desde cuero, lana, toalla, terciopelo... cada uno
    con unas propiedades opticas y mec\'anicas diferentes y \'unicas que influyen completamente en las
    decisiones de dise\~no del producto.

\section{Author: aplicaci\'on web donde el usuario dise\~na sus creaciones}
    El proceso empieza con la creaci\'on de un digital twin de un hilo, capturando las propiedades
    f\'isicas y mec\'anicas de las fibras que lo componen y que sirven como entradas a los algoritmos de
    simulaci\'on y render. A partir de este hilo los usuarios pueden dise\~nar un tejido o escoger entre
    una gran cantidad de ellos, en funci\'on de la frecuencia o el tipo de entrelazado de los hilos.
    Utilizando estos tejidos se dise\~nan patrones en plano con las formas de las piezas que conforman
    una prenda y se permite al usuario previsualizar y editar la construcci\'on del modelo en una vista
    3D sobre un maniqu\'i. Una vez completado el ciclo de diseño se generan autom\'aticamente fichas técnicas
    con especificaciones sobre los elementos est\'eticos o constructivos que facilitan la comunicaci\'on
    entre todas las partes implicadas en el proceso de fabricaci\'on.
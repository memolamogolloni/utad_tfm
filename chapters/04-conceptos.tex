\chapter{Conceptos previos}

\section{Render}
    El render es un area de la computacion gr\'afica que se encarga de traducir estructuras de datos y funciones a p\'ixeles en la pantalla
    con el objetivo de formar una imagen bidimensional.
    
    La representaci\'on digital de una imagen consiste en un grafo que contiene los elementos de la escena: c\'amara, luces y objetos visibles en
    la escena. Sobre esta escena virtual, se simulan las interacciones de la luz con las superficies para acabar estimando un color para cada
    superficie visible desde el punto de vista de la c\'amara.

    Sus principales aplicaciones se encuentran en industrias como el cine de animaci\'on, los efectos especiales,
    videojuegos, arquitectura, simuladores del \'ambito industrial, visualizaciones de producto o la im\'agen m\'edica.
    Debido a las diferentes necesidades y caracteristicas de cada industria, podemos categorizar a los motores de render
    principalemente en base a dos criterios: su intenci\'on de imitar o no la realidad (motores fotorrealistas y no
    fotorrealistas) y la necesidad o no de interaccion de usuario (motores online y offline).

    En este trabajo, la intencion es el renderizado de telas con el mayor grado de fidelidad posible a la realidad en una
    aplicaci\'on con interaccion de usuario, por lo que nos centraremos en el renderizado online fotorrealista.

    \todo[inline] {
        A\~nadir informaci\~on elementos y grafo de escena. 
        A scene file contains objects in a strictly defined language or data structure; it would contain geometry, viewpoint, texture, lighting, and shading information as a description of the virtual scene.
        The data contained in the scene file is then passed to a rendering program to be processed and output to a digital image or raster graphics image file. 
        If a scene is to look relatively realistic and predictable under virtual lighting, the rendering software should solve the rendering equation
    }

\section{Modelo f\'isico de la luz}
    Para conseguir im\'agenes parecidas a la realidad, el render utiliza algoritmos que simulan el comportamiento f\'isico de la luz.
    El comportamiento de los fotones se describe con el fenomeno cuantico conocido como dualidad onda-part\'icula. El comportamiento
    como particula establece que se mueve en linea recta, rebotando con los objetos del entorno a una velocidad de 300.000 km/s,
    la mas alta conocida y que en el modelo de simulaci\'on se considera infinita. Por otra parte, el comportamiento de de la luz como
    onda, permite trabajar sobre propiedades como frecuencia, amplitud y longitud de onda, que se utilizan para modelar la representaci\'on
    digital del color .
    \todo[inline] {
        Inverse square law. Ley de Lambert. Ley de reflexion. Ley de Snell
    }

\section{Color}
    La interpretaci\'on del color depende de dos cosas, por una parte, las longitudes de onda que inciden sobre una superficie, y como
    esta superficie absorve o refleja esas longitudes de onda y por otra parte, como nuestros receptores reaccionan al espectro visible.
    
    La fotometr\'ia estudia la percepci\'on de la luz por el ojo humano, mientras que la radiometr\'ia se encarga del estudio del fen\'omeno
    f\'isico y es en la que se centra el render para la simulacion f\'isica de la escena tridimensional.
    Para medir la luz se utiliza el espectro, una gr\'afica cartesiana en la que se mide sobre el eje x la longitud de onda y en el eje y,
    su amplitud. El rango de longitudes de onda que se utiliza en render esta limitado a las longitudes de onda visibles por el ojo humano,
    que se conoce como espectro visible. Las ondas de luz pueden sumarse, como se suman las senhales y dar como resultado un nuevo color.
    
    En las aplicaciones de software, la representacion del espectro completo es muy costoso, por lo que se suele trabajar con una
    simplificacion del modelo, la combinacion de espectros base o la descomposicion en funciones base.
    Una de los representaciones mas populares es el modelo RGB, que se corresponde con los tres tipos de conos del ojo humano, los mas
    sensibles al verde, al rojo y al azul.  Es un modelo aditivo, por lo que cada color se obtiene a partir de la suma de estos tres. Aunque
    existen otras representaciones digitales del color, dependiendo de su aplicaci\'on, CMYK, XYZ, etc. el RGB es el mas comun en el mundo de
    la computacion gr\'afica.

\section{Unidades de medida de la luz}
    Cuando hablamos de la cantidad de luz, no es la medicion sobre un foton de luz independiente, si no que se tratan de medidas en relacion
    al tiempo, la direccion o el \'area.

    \subsection{\'Angulo s\'olido}

    \subsection{Flujo}
        La unidad que representa la energia sobre la unidad de tiempo es el flujo, representado por Phi. Es la energía que transportan las ondas
        por unidad de tiempo, se mide en vatios. En los motores de render se utiliza para expresar la cantidad total de energia emitida por un a fuente de luz.
        \begin{equation}
            \Phi_e = \dfrac{d{Q_e}}{dt}
        \end{equation}
        \singlespacing

    \subsection{Intensidad}
        \begin{equation}
            I = \dfrac{d\Phi}{d\omega}
        \end{equation}
        \singlespacing

    \subsection{Irradiancia}
        La irradiancia, es la cantidad de energ\'ia por unidad de tiempo por unidad de superficie, o flujo por superficie. Se representa como
        E, su unidad son los vatios/m2 y se utiliza para medir la cantidad de luz que incide sobre una superficie.
        \begin{equation}
            E = \dfrac{d\Phi}{dA}
        \end{equation}
        Cuando este flujo de radiancia se mide en direcci\'on contraria, de salida, se llama emitancia (M)
        \singlespacing

    \subsection{Radiancia}
        \begin{equation}
            L = \dfrac{d^2\Phi}{dA_{proj}d\omega}
        \end{equation}


\section{Ecuaci\'on de render}
    El prop\'osito de la ecuacion de render es conocer el valor de radiancia que llega a la camara en una direcci\'on
    por cada pixel  de la camara.

    Para saber la radiancia sobre un punto en una direcci\'on, utilizamos la ecuacion de reflectancia, que depende
    de la luz que llega al puntos, el coseno del angulo con el que incide la luz y el BRDF (bidirectional reflectance
    distribution function), que modela el comportamiento de la luz al rebotar sobre la superficie.
    \begin{equation}
        L(x\xrightarrow{}{\vec{v}{\,}})
        = L_i(x\xleftarrow{}{\vec{l}{\,}})
        f(x, \vec{v}{\,}, \vec{l}{\,})
        (\vec{l}{\,}\cdotp{\vec{n}{\,}})
    \end{equation}
    \singlespacing

    Ademas de recibir la luz directamente de una fuente de luz, el punto tambi\'en puede recibir luz rebotada por otras
    partes del entorno:
    \begin{equation}
        L(x\xrightarrow{}{\vec{v}{\,}})
        = L_e(x, \vec{v}{\,}) +
        L_i(x\xleftarrow{}{\vec{l}{\,}})
        f(x, \vec{v}{\,}, \vec{l}{\,})
        (\vec{l}{\,}\cdotp{\vec{n}{\,}})
    \end{equation}
    \singlespacing

    Podr\'iamos sumar todas las fuentes de luz para calcular la radiancia total debido a las fuentes de luz:
    \begin{equation}
        L(x\xrightarrow{}{\vec{v}{\,}})
        = L_e(x, \vec{v}{\,}) +
        \sum{}L_i(x\xleftarrow{}{\vec{l_i}{\,}})
        f(x, \vec{v}{\,}, \vec{l_i}{\,})
        (\vec{l_i}{\,}\cdotp{\vec{n}{\,}})
    \end{equation}
    \singlespacing

    Sin embargo, para tener en cuenta el total de luz incidente, ademas de las fuentes de luz (luz directa), debemos
    de tener en cuenta toda la luz proviniente del entorno en todas direcciones (luz indirecta):
    \begin{equation}
        L(x\xrightarrow{}{\vec{v}{\,}})
        = L_e(x, \vec{v}{\,}) +
        \int_{\Omega}L_i(x'\xrightarrow{}{\vec{l_i}{\,}})
        f(x, \vec{v}{\,}, \omega_i)
        \cos\theta_id\omega_i
    \end{equation}
    \singlespacing

    Como podemos ver, la radiancia aparece a los dos lados de la ecuaci\'on, lo que quiere decir que para calcular la radiancia
    en un punto, necesitamos la radiancia en otros puntos por lo que se trata de un calculo recursivo.

    Esta ecuaci\'on cumple dos propiedades: linearidad y descomposicion por partes, gracias a ellas, podemos transformar el problema
    en una serie de Neumann.

    Agrupando los terminos conseguimos:
    \begin{equation}
        L_{out} = \Upsilon L_{in}\\
        L_{in} = \Lambda L_{out}
    \end{equation}
    \singlespacing

    Y juntandolos:
    \begin{equation}
        L_{out} = \Upsilon\Lambda L_{out} + L_e
    \end{equation}
    \singlespacing

    Al agruparlos:
    \begin{equation}
        L_{out} = K L_{out} + L_e
    \end{equation}
    \singlespacing

    Reordenarlos:
    \begin{equation}
        (I-K)L_{out} = L_e\\
        L_{out} = (I - K)^{-1}L_e
    \end{equation}
    \singlespacing

    Se puede separar la emisi\'on en partes:
    \begin{equation}
        L_{out} = (I - K)^{-1}E_1 + (I-K)^{-1}E_2
    \end{equation}
    \singlespacing

    Y sustituyendo:
    \begin{equation}
        (I - K)^{-1} = \sum_{i=0}^{\infty}K^i = I + K + KK + KKK + K + \dotsi
    \end{equation}
    \singlespacing

    Obtenemos la serie de Neumann:
    \begin{equation}
        L_{out} = L_e + KL_e + KKL_e + KKKL_e + \dotsi
    \end{equation}
    \singlespacing

    Demostrando que la radiancia total puede ser calculada como la suma de luz emitida, mas la suma de la reflejada una vez,
    mas la suma de la reflejada dos veces, etc.


\section{Iluminaci\'on}
    Como hemos visto en la ecuaci\'on de render, para calcular el total de radiancia proviniente de una superficie, en funci\'on de
    su origen, podemos descomponer la suma de su iluminaci\'on en la suma de iluminaci\'on directa e iluminaci\'on indirecta.
    \subsection{Iluminaci\'on local o directa}
        Es la que proviene directamente de una fuente de luz y que contribuye en gran parte al aspecto final de una escena. Tendremos
        en cuenta la luz que viaja de una fuente de luz a una superficie y de la superficie al ojo.
    \subsection{Iluminaci\'on global o indirecta}
    \subsection{Componentes de la luz: difuso, especular y ambiente}

\section{PBR}
    PBR o \textit{physically based rendering} es el termino que se utiliza para nombrar los algoritmos de render basados en un modelo f\'isico.
    La intenci\'on es conseguir aproximaciones r\'apidas y plausibles de la interacci\'on de un flujo de luz con una superficie. Las
    principales ventajas de este tipo de algoritmos son la consistencia bajo diferentes condiciones de luz y su manejo intuitivo por
    parte de los artistas.

    Un motor PBR ha de tener en cuenta las leyes de las f\'isica para el tratamiento de todos los elementos de la escena: luces, camaras
    y superficies de los objetos visibles.

    \subsection{C\'amara}
    La camara se utiliza para determinar la parte de la escena visible en la imagen 2D final.

    Una camara de un motor de render PBR utiliza la matriz de proyeccion perspectiva para simular la
    vista del ojo humano y su modelo b\'asico consiste en la simulaci\'on de una camara estenopeica, un modelo basico de c\'amara fotogr\'afica que
    consiste en un compartimento oscuro, con un fondo sobre el que se coloca el negativo fotogr\'afico, mientras que en la parte delantera cuenta
    con una perforaci\'on, por la que entra la luz antes de golpear con el material fotosensible.

    La c\'amara permite manipular aspectos esenciales de la imagen final como el campo de visi\'on, la distancia m\'inima y m\'axima de la escena
    con respecto al ojo. Adem\'as, en funci\'on de los objetivos del motor de render, puede simular efectos de las camaras fotogr\'aficas, como la
    profunidad de campo, la velocidad de obturaci\'on o el tiempo de apertura, asi como su posici\'on y orientaci\'on, que son, junto a la posici\'on
    de la luz, factores clave para obtener la radiancia de una superficie.

    \subsection{Luces}
        Gran parte de la radiancia total que se ve sobre un punto suele venir dado por las fuentes de iluminaci\'on directa. En una escena digital,
        las luces son aprominaciones a las fuentes de luz que tenemos en la naturaleza.
        
        Podemos distinguir como los tipos de luz prinpicales:

        \subsubsection{Luz de ambiente}
            Luz que simula la componente de iluminaci\'on global en caso de que esta no se calcule. Suma un valor fijo a la iluminacion de cada punto.
        \subsubsection{Luz puntual}
            Luz emitida desde un punto en todas direcciones.
        \subsubsection{Luz de foco}
            Emite desde un punto en un cono orientado.
        \subsubsection{Luz direccional}
            Simula luz tan lejana que sus rayos son paralelos entre si, como por ejemplo el sol.
        \subsubsection{Luz de area}
            Luz emitida desde una superficie, normalmente un plano o un disco
        \subsubsection{Luz de entorno}
            Usa un mapa sobre una esfera para simular las condiciones de iluminacion de la imagen

    \subsection{Objetos}
        Los objetos de la escena est\'an compuestos por su geometr\'ia y su material.

        La geometr\'ia es una secuencia de v\'ertices en el espacio que conforman pol\'igonos que describen la forma de un objeto. La informaci\'on de la
        geometr\'ia se almacenan en estructuras de datos que ademas de la informaci\'on de v\'ertices, puede contener informaci\'on sobre las caras de los
        pol\'igonos, las normales, tangentes, etc.

        El material, por otra parte, describe las caracter\'isticas visuales de la superficie del objeto. En la naturaleza, adem\'as del color de un objeto,
        podemos percibir propiedades de la textura, su conductividad, dielectrico o metalico, su rugosidad o su reflectividad a trav\'es de la interacci\'on
        de la luz con la superficie de un objeto. Esta interacci\'on a nivel microscopico entre la luz y la superficie de un material, se describe con una
        ecuacion conocida como BSDF.


\section{BxDF}
    Son funciones que permiten describir el comportamiento de la luz al golpear una superficie. Para ello toma en cuenta el \'angulo de incidencia del
    rayo de luz sobre la superficie y el rayo de salida, reflejado o refractado y su resultado es el ratio entre el \'angulo del rayo y la superficie y
    la intesidad de salida de la luz.

    A continuaci\'on se detallan los nombres de las funciones en funcion del fen\'omeno f\'isico que modelan.

    \subsection{BRDF}
        Es la funci\'on que modela el comportamiento de la luz al golpear una superficie opaca, la reflexi\'on. Fue definido por primera vez en 1965 por
        Fred Nicodemus y su definici\'on es el ratio entre radiancia reflectada e irraciancia incidente.
        Para determinar el \'angulo de salida del rayo se utiliza la ley de la reflexi\'on y las tres caracter\'isticas que ha de cumplir un BRDF basado en
        f\'isica es que sea positivo, que cumpla con la reciprocidad de Helmholtz y que cumpla con la ley de conservaci\'on de la energ\'ia.

    \subsection{BTDF}
        Describe el comportamiento del rayo de luz al atravesar una superficie, la refracci\'on. Para calcular el angulo de salida del rayo se utiliza la
        ley de Snell, y al contrario que el BRDF, no sumple el principio de reciprocidad de Helmholtz.

    \subsection{BSSRDF y BSSTF}
        Son ampliaciones del modelo de reflexi\'on y refracci\'on, respectivamente, teniendo en cuenta las reflexiones internas del rayo a traves de la
        superficie del objeto.

    \subsection{BSDF}
        Se utiliza comunmente para hablar de cualquier forma de BxDF. En un sentido mas estricto, se refiere al conjunto de un BSSRDF y un BSSTDF.

\section{T\'erminos del BRDF}

\section{Teor\'ia de microfacetas}

\section{Render offline y online}

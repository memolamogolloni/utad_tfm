\chapter{Marco colaborativo}
\todo[inline]{Cambiar/mejorar}
Seddi es una startup nacida como fruto de la investigacion en tecnologias de simulacion optica y mecanica aplicadas al sector textil.
Propone ofrecer una solucion digital que permita a los disenhadores, patronistas o comerciales trabajar en un entorno colaborativo que
representa con gran fidelidad los elementos constructivos de un prenda de ropa: tejido, costuras, dobladillos, etc.

La caida, el brillo y, en general, la apariencia de una prenda son factores clave para validar o rechazar el producto, estas caracteristicas
dependen de propiedades opticas y mecanicas que las herramientas de disenho actuales no tienen en cuenta, por lo que los profesionales del
sector no satisfacen las necesidades de los profesionales de la industria, que en la mayor parte de los casos mantienen un proceso de
trabajo artesanal en el que el prototipado y construccion de la prenda es un proceso iterativo entre patronista y disenhador y cuyo
resultado depende en gran medida de la experiencia y destreza de estos profesionales.\\

Para ofrecer su solucion, Seddi cuenta con departamentos de investigacion simulacion y captura tanto mecanica como optica, que desarrollan
nuevos algoritmos y hardware que permitan un mayor grado de fidelidad en el momento de captura de los parametros que conforman la apariencia
de una prenda, tanto como mejorar su representacion digital asi como departamentos de desarrollo de producto, se encargan del proceso de
captura utilizando estas tecnologias propietarias asi como de integrar estos algoritmos en diferentes plataformas que permitan al consumidor
final interactuar sobre los tejidos digitales durante diferentes etapas asociadas a este proceso de produccion con el fin de facilitar,
acelerar y reducir los costes de produccion.

Para conseguir un resultado fidedigno, en Seddi, el proceso comienza creando un clon digital del tejido, la obtencion de parametros opticos y
mecanicos que identifican el tejido y se valida utilizando las tecnologias de simulacion y render de la empresa. 

Todos las herramientas de Seddi se alojan en la nube, lo que garantiza disponibilidad inmediata a los recursos a todo el equipo involucrado en
el proceso de produccion. Estos equipos, con un negocio de la moda cada dia mas globalizado, pueden tratarse de equipos multidisciplinares de
diferentes empresas o paises, por lo que es primordial un flujo de comunicacion constante que minimice los errores y los costes economicos y
medioambientales de los flujos de disenho y prototipado

Gracias a esta estrategia de implementación, el contenido diseñado está disponible de manera inmediata y editable en un escenario global. En la
actualidad, el negocio de la moda está completamente globalizado, y las tareas de diseño y prototipado de muchas marcas se hacen entre equipos
internacionales o mediante subcontratación entre empresas. La  utilización de las soluciones cloud de Seddi permitirá a los equipos colaborar remotamente sobre un mismo diseño.
En Seddi, el proceso para llegar a construir una prenda, comienza con la digitalizacion de un tejido, que se consigue a partir de muestras que
son analizadas por maquinas de captura desarrolladas en Seddi, para obtener los parametros mecanicos y opticos

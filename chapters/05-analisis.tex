\chapter{An\'alisis}

\section{Motores de render: online y offline}
    \subsection{Online}
    \subsection{Offline}

\section[Plataformas de render: OpenGL y WebGL]{Plataformas de render: OpenGL \\y WebGL}
    \subsection{OpenGL 4.0}
    \subsection{WebGL 2.0}

\section{C\'alculo de la radiancia}
    \subsection{Iluminaci\'on directa}
    \subsection{Iluminaci\'on indirecta}
    \subsection{T\'erminos del BRDF}

\section{Comparaci\'on ThreeJS - Disney 2012}
    \subsection{Esquema BRDF ThreeJS}
        En el modelo de Disney 2012, se utilizan dos BRDFs distintos, uno objetos metálicos y
        otro para dieléctricos y se hace blending entre ellos:
    \singlespacing
    \begin{lstlisting}[caption=My Javascript Example]
// main() {
// ...

float metalnessFactor = metalness;
#ifdef USE_METALNESSMAP
    vec4 texelMetalness = texture2D( metalnessMap, vUv );
    metalnessFactor *= texelMetalness.b;
#endif

// ...

material.diffuseColor = diffuseColor.rgb * (1.0 - metalnessFactor);
material.specularRoughness = clamp(roughnessFactor, 0.04, 1.0);

#ifdef REFLECTIVITY
    // MAXIMUM_SPECULAR_COEFFICIENT = 0.16
    material.specularColor = mix(vec3(MAXIMUM_SPECULAR_COEFFICIENT * pow2(reflectivity), diffuseColor.rgb, metalnessFactor);
#else
    // DEFAULT_SPECULAR_COEFFICIENT = 0.04
    material.specularColor = mix(vec3(DEFAULT_SPECULAR_COEFFICIENT), diffuseColor.rgb, metalnessFactor);

// ...
// calculate direct illumination...
// calculate indirect illumination...
// ...
                                 
// }
    \end{lstlisting}

    \subsection{Iluminaci\'on directa}
        El valor del difuso se calcula directamente con el BRDF de Lambert.

        Para calcular el especular, se acumulan las contribuciones de los tres
        lóbulos. Si se utiliza clearcoat, se estima el *clearcoatDHR*
        (clearcoat directional hemispherical reflectivity) y se calcula la
        contribución al especular del clearcoat. Al siguiente lóbulo, *sheen* en
        caso de existir, o si no directamente al BRDF del material base, se le
        aplica un factor de 1.0 - clearcoatDHR y se acumula su valor.
        \singlespacing
        \begin{lstlisting}[caption=My Javascript Example]
// main() {
// ...

float metalnessFactor = metalness;
#ifdef USE_METALNESSMAP
    vec4 texelMetalness = texture2D( metalnessMap, vUv );
    metalnessFactor *= texelMetalness.b;
#endif

// ...

material.diffuseColor = diffuseColor.rgb * ( 1.0 - metalnessFactor );
material.specularRoughness = clamp( roughnessFactor, 0.04, 1.0 );

#ifdef REFLECTIVITY
    // MAXIMUM_SPECULAR_COEFFICIENT = 0.16
    material.specularColor = mix(vec3( MAXIMUM_SPECULAR_COEFFICIENT * pow2(reflectivity),                              diffuseColor.rgb, metalnessFactor );
#else
    // DEFAULT_SPECULAR_COEFFICIENT = 0.04
    material.specularColor = mix(vec3(DEFAULT_SPECULAR_COEFFICIENT), diffuseColor.rgb,                                metalnessFactor );

// ...
// calculate direct illumination...
// calculate indirect illumination...
// ...
                                    
// }
        \end{lstlisting}

        \subsubsection{Difuso}
            El difuso de Disney hace un blending entre el modelo de difuso de base y el BRDF para
            subsurfaces de HanrahanKrueger.
            $$
            f_d = \frac{baseColor}{\pi}(1 + (F_{D90} - 1) (1 - cos{\theta}_t)^5)(1 + F_{D90} - 1)
            (1 - cos\theta_v)^5)
            $$
            La implementación de ThreeJS utiliza el BRDF para el difuso de Lambert:
            \singlespacing
            \begin{lstlisting}[caption=My Javascript Example]
vec3 BRDF_Diffuse_Lambert( const in vec3 diffuseColor ) {
    return RECIPROCAL_PI * diffuseColor;
}
            \end{lstlisting}

        \subsubsection{L\'obulo primario (material base)}
            Representa el material base y puede ser anisotr\'opico y/o  met\'alico.

            Implementaci\'on en ThreeJS:
            \singlespacing
            \begin{lstlisting}[caption=My Javascript Example]
vec3 BRDF_Specular_GGX(
    const in IncidentLight incidentLight,
    const in vec3 viewDir,
    const in vec3 normal,
    const in vec3 specularColor,
    const in float,
    roughness
) {
    float alpha = pow2(roughness);
    vec3 halfDir = normalize(incidentLight.direction + viewDir);
    float dotNL = saturate(dot(normal, incidentLight.direction));
    float dotNV = saturate(dot(normal, viewDir));
    float dotNH = saturate(dot(normal, halfDir));
    float dotLH = saturate(dot(incidentLight.direction, halfDir));
    vec3 F = F_Schlick(specularColor, dotLH);
    float G = G_GGX_SmithCorrelated(alpha, dotNL, dotNV);
    float D = D_GGX(alpha, dotNH);
    return F * (G * D);
}
            \end{lstlisting}
            \singlespacing
            El BRDF de ThreeJS es isotrópico y no metálico.
            \singlespacing
            T\'ERMINOS DEL BRDF:
            \singlespacing
            \begin{enumerate}
            \item D
                El mismo que Disney 2012 \autocite{disney12}
            $$D_{GGX}(m) = \frac{\alpha^2}{\pi((n\cdotp{m^2})(\alpha^2 - 1 ) + 1)^2}$$
            \singlespacing
            \begin{lstlisting}[caption=My Javascript Example]
float D_GGX( const in float alpha, const in float dotNH ) {
    float a2 = pow2( alpha );
    float denom = pow2( dotNH ) * ( a2 - 1.0 ) + 1.0;
    return RECIPROCAL_PI * a2 / pow2( denom );
}
            \end{lstlisting}
            \singlespacing
            \item F
                Versi\'on optimizada de Fresnel-Schlick, la misma que la presentadapor Epic en
                el SIGGRAPH de 2013\autocite{unreal}
            \singlespacing
            \begin{lstlisting}[caption=My Javascript Example]
vec3 F_Schlick( const in vec3 specularColor, const in float dotLH ) {
    float fresnel = exp2( ( -5.55473 * dotLH - 6.98316 ) * dotLH );
    return ( 1.0 - specularColor ) * fresnel + specularColor;
}
            \end{lstlisting}
            \singlespacing
            \item G
                Utiliza que la misma presentada para Frostbite en el SIGGRAPH de 2014\autocite{frostbite}
            \singlespacing
            \begin{lstlisting}[caption=My Javascript Example]
float G_GGX_SmithCorrelated( const in float alpha, const in float dotNL, const in float dotNV ) {
    float a2 = pow2( alpha );
    float gv = dotNL * sqrt( a2 + ( 1.0 - a2 ) * pow2( dotNV ) );
    float gl = dotNV * sqrt( a2 + ( 1.0 - a2 ) * pow2( dotNL ) );
    return 0.5 / max( gv + gl, EPSILON );
}
            \end{lstlisting}
            \singlespacing
            \end{enumerate}

        \subsubsection{L\'obulo secundario (clearcoat)}
            El l\'obulo secundario representa una capa de clearcoat sobre el material y siempre es isotr\'opica
            y no met\'alica. Se utiliza un IOR fijo de 1.5, que representa la refracci\'on del poliuretano (F0 = 0.04)
            ThreeJS utiliza el mismo BRDF que para el material base, con un F0 = 0.04
            \singlespacing
            \begin{lstlisting}[caption=My Javascript Example]
float clearcoatDHR = material.clearcoat * clearcoatDHRApprox(material.clearcoatRoughness, ccDotNL);
// IOR = 1.5 -> F0 = DEFAULT_SPECULAR_COEFFICIENT = 0.04
reflectedLight.directSpecular += ccIrradiance * material.clearcoat * BRDF_Specular_GGX(
    directLight,
    geometry.viewDir,
    geometry.clearcoatNormal,
    vec3(DEFAULT_SPECULAR_COEFFICIENT),
    material.clearcoatRoughness
);
            \end{lstlisting}
            \singlespacing
            Mismo BRDF que el l\'obulo primario:
            \singlespacing
            \begin{lstlisting}[caption=My Javascript Example]
vec3 BRDF_Specular_GGX(const in IncidentLight incidentLight, const in vec3 viewDir,                              const in vec3 normal, const in vec3 specularColor, const in float                        roughness ) {
    float alpha = pow2( roughness );
    vec3 halfDir = normalize( incidentLight.direction + viewDir );
    float dotNL = saturate( dot( normal, incidentLight.direction ) );
    float dotNV = saturate( dot( normal, viewDir ) );
    float dotNH = saturate( dot( normal, halfDir ) );
    float dotLH = saturate( dot( incidentLight.direction, halfDir ) );
    vec3 F = F_Schlick( specularColor, dotLH );
    float G = G_GGX_SmithCorrelated( alpha, dotNL, dotNV );
    float D = D_GGX( alpha, dotNH );
    return F * ( G * D );
}
            \end{lstlisting}
            \singlespacing
            Calcula el factor de reflectividad del clearcoat para aplicarle
            1 - clearcoatDHR al siguiente l\'obulo

        \subsubsection{L\'obulo secundario (clearcoat)}
            El l\'obulo secundario representa una capa de clearcoat sobre el material y
            siempre es isotr\'opica y no met\'alica.
            Se utiliza un IOR fijo de 1.5, que representa la refracci\'on del poliuretano (F0 = 0.04)
        
            ThreeJS utiliza el mismo BRDF que para el material base, con un F0 = 0.04
            \singlespacing
            \begin{lstlisting}[caption=My Javascript Example]
float clearcoatDHR = material.clearcoat * clearcoatDHRApprox(material.clearcoatRoughness,                      ccDotNL );
// IOR = 1.5 -> F0 = DEFAULT_SPECULAR_COEFFICIENT = 0.04
reflectedLight.directSpecular += ccIrradiance * material.clearcoat * BRDF_Specular_GGX(
        directLight,
        geometry.viewDir,
        geometry.clearcoatNormal,
        vec3(DEFAULT_SPECULAR_COEFFICIENT
    ),
    material.clearcoatRoughness
);
            \end{lstlisting}
            \singlespacing
            Mismo BRDF que el lóbulo primario
            \singlespacing
            \begin{lstlisting}[caption=My Javascript Example]
vec3 BRDF_Specular_GGX(
    const in IncidentLight incidentLight,
    const in vec3 viewDir,
    const in vec3 normal,
    const in vec3 specularColor,
    const in float roughness
) {
    float alpha = pow2( roughness );
    vec3 halfDir = normalize( incidentLight.direction + viewDir );
    float dotNL = saturate( dot( normal, incidentLight.direction ) );
    float dotNV = saturate( dot( normal, viewDir ) );
    float dotNH = saturate( dot( normal, halfDir ) );
    float dotLH = saturate( dot( incidentLight.direction, halfDir ) );
    vec3 F = F_Schlick( specularColor, dotLH );
    float G = G_GGX_SmithCorrelated( alpha, dotNL, dotNV );
    float D = D_GGX( alpha, dotNH );
    return F * ( G * D );
}
            \end{lstlisting}
            \singlespacing
            Calcula el factor de reflectividad del clearcoat para aplicarle *1 - clearcoatDHR* al
            siguiente lóbulo
            \singlespacing
            \begin{lstlisting}[caption=My Javascript Example]
// Clearcoat directional hemispherical reflectivity
float clearcoatDHRApprox( const in float roughness, const in float dotNL ) {
    // DEFAULT_SPECULAR_COEFFICIENT = 0.04
    return DEFAULT_SPECULAR_COEFFICIENT + ( 1.0 - DEFAULT_SPECULAR_COEFFICIENT ) *
            (pow( 1.0 - dotNL, 5.0 ) * pow( 1.0 - roughness, 2.0 ) );
}
            \end{lstlisting}
            \singlespacing

        \subsubsection{L\'obulo terciario (sheen)}
            El l\'obulo terciario se utiliza para representar el sheen, que aporta un extra de
            reflectancia en los \'angulos muy agudos.

            Este componente se parece mucho al factor de Fresnel y para modelar su BRDF se
            utiliza Schlick-Fresnel: \(sheen * (1 − cos{\zeta}d\exp{5})\),  se puede te\~nir su
            color con uno diferente al del material base con el par\'ametro sheenTint.
            
            ThreeJS implementation
            \singlespacing
            \begin{lstlisting}[caption=My Javascript Example]
// Estevez and Kulla 2017, "Production Friendly Microfacet Sheen BRDF"
float D_Charlie(float roughness, float NoH) {
    float invAlpha = 1.0 / roughness;
    float cos2h = NoH * NoH;
    float sin2h = max(1.0 - cos2h, 0.0078125);
    return (2.0 + invAlpha) * pow(sin2h, invAlpha * 0.5) / (2.0 * PI);
}

// Neubelt and Pettineo 2013, "Crafting a Next-gen Material Pipeline for The Order: 1886"
float V_Neubelt(float NoV, float NoL) {
    return saturate(1.0 / (4.0 * (NoL + NoV - NoL * NoV)));
}

vec3 BRDF_Specular_Sheen( const in float roughness, const in vec3 L, const in GeometricContext geometry, vec3 specularColor ) {
    vec3 N = geometry.normal;
    vec3 V = geometry.viewDir;
    vec3 H = normalize( V + L );
    float dotNH = saturate( dot( N, H ) );
    return specularColor * D_Charlie(roughness, dotNH) * V_Neubelt(dot(N, V), dot(N, L));
}
            \end{lstlisting}
            \singlespacing


    \subsection{Iluminaci\'on indirecta}

\newpage
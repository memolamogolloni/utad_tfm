\chapter{Introducci\'on}

\section{Motivaci\'on}

\section{Objetivos}
\todo[inline]{ToDo}
Objetivos generales:
\begin{enumerate}
    \item Conseguir materiales m\'as realistas para la visualizaci\'on tejidos sobre el navegador web.
    \item Aumentar coherencia entre las im\'agenes del renderer online y el offline en Seddi.
\end{enumerate}

Objetivos espec\'ificos
\begin{enumerate}
    \item Analizar las t\'ecnicas empleadas en los motores de render de Seddi. 
    \item Identificar las diferencias en los algoritmos que m\'as afectan al resultado visual.
    \item Extender el motor de render del cliente para mostrar con mayor fidelidad ciertas propiedades
          de los tejidos.
  \end{enumerate}

\section{Contribuciones}

\section{Gu\'ia de la memoria}

\textbf{Cap\'itulo 1. Introducci\'on}
En este cap\'itulo se explican brevemente los objetivos y motivaciones del proyecto, as\'i como las aportaciones
de este trabajo.\\

\textbf{Cap\'itulo 2. Marco colaborativo}
En el cap\'itulo 2 se explica la relaci\'on de colaboraci\'on del presente proyecto, que se desarrolla bajo el amparo de Seddi.
Se contextualiza el marco de desarrollo, explicando los objetivos y motivaciones de la empresa.\\

\textbf{Cap\'itulo 3. Estado del arte}
El tercer cap\'itulo se ofrece una visi\'on del estado del arte en cuanto a soluciones en la nube similares a los servicios
de Seddi. Adem\'as se listan los trabajos sobre materiales e iluminaci\'on en tiempo real que guardan relaci\'on con la
soluci\'on presentada en este proyecto.\\

\textbf{Cap\'itulo 4. PBR}
En este cap\'itulo se explican las t\'ecnicas est\'andar en la actualidad para la representaci\'on de materiales realistas y
sirve como base te\'orica para la propuesta presentada en el cap\'itulo 6.\\

\textbf{Cap\'itulo 5. Infraestructura y servicios web}
El cap\'itulo 5 explica la integraci\'on de la propuesta sobre los servicios en la nube de Seddi y la integraci\'on de un nuevo material
sobre la librer\'ia de WebgGL ThreeJs.\\

\textbf{Cap\'itulo 6. Modelos de iluminaci\'on para tejidos}
En este cap\'itulo se explica en detalle el modelo de shading elegido para integrar sobre el cliente de Author en Seddi, as\'i como
la integraci\'on de este nuevo material.\\

\textbf{Cap\'itulo 7. Resultados}
Presentaci\'on de los resultados obtenidos y comparaci\'on del nuevo modelo frente a los modelos est\'andar PBR que ofrece ThreeJs.\\

\textbf{Cap\'itulo 8. Discusi\'on y trabajo futuro}
Comentarios sobre el planteamiento, desarrollo, resultados obtenidos y posibles aplicaciones. Tambi\'en se establecen
l\'ineas de trabajo futuro 

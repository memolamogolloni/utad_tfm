\chapter{PBR}
PBR o \textit{physically based rendering} es el t\'ermino que se utiliza para nombrar al grupo de t\'ecnicas basados en un
modelo f\'isico. La intenci\'on es conseguir aproximaciones r\'apidas y plausibles de la interacci\'on de un flujo de luz
con una superficie. Las principales ventajas de este tipo de algoritmos son la consistencia bajo diferentes condiciones de
luz y su manejo intuitivo por parte de los artistas.
Los motores PBR cumplen con la ley de la conservaci\'on de la energ\'ia y utilizan BRDFs basados en la teor\'ia de
microfacetas. Adicionalmente, otros elementos de la escena, como la c\'amara o las luces, pueden estar basados en modelos
f\'isicos para ofreciendo para aumentar el grado de realismo.

\section{Ecuaci\'on de render}
El prop\'osito de la ecuacion de render es conocer el valor de radiancia que llega a la camara en una direcci\'on
por cada pixel  de la camara.

Para saber la radiancia sobre un punto en una direcci\'on, utilizamos la ecuacion de reflectancia, que depende
de la luz que llega al puntos, el coseno del angulo con el que incide la luz y el BRDF (bidirectional reflectance
distribution function), que modela el comportamiento de la luz al rebotar sobre la superficie.\\

\begin{eqfloat}[!htb]
    \begin{equation}
        L_o(p, w_o) = L_e(x, w_o) + \int_\Omega{f_r(p, w_i, w_o) L_i(p, w_i) n\cdot{w_i}dw_i}
    \end{equation}
  \caption{Ecuaci\'on de render}
\end{eqfloat}
\singlespacing

Siendo $L_o(p, w_o)$ la luz reflectada por la superficie, $L_e(x, w_o)$ la luz emitida por la superficie, que est\'a fuera
de la integral porque es independiente de la luz reflejada o refractada por la superficie. $\int_{\Omega}[...]dw$ que representa
que la operaci\'on se repite para cada \'angulo s\'olido dentro de la semiesfera $f_r(p, w_i, w_o)$ el BRDF $L_i(p, w_i)$ la luz
que llega al punto de la superficie que estamos evaluando $n\cdot{w_i}$ el factor de atenuaci\'on de la luz con respecto a la normal.

\section{BRDF}

El BRDF, o función de ditribuci\'on bidireccional de reflectividad, es la parte de la ecuaci\'on que describe el
comportamiento de la luz al golpear sobre la superficie y se utiliza para rmodelar propiedades f\'isicas de los diferentes
materiales. Es una funcion estad\'istica que calcula cuanta de la luz que incide sobre un material es reflejada en direcci\'on
a la c\'amara, o, en otras palabras, para ello toma como argumentos la direccion de la luz, $w_i$, y la dirección de salida,
$w_o$, la normal de la superficie, n, y un par\'ametro $\alpha$ que representa la rugosidad del material.

\begin{figure}[H]
    \vspace{0.5cm}
    \centering
      \frame{\includegraphics[scale=0.7]{microfacet-diagram}}
    \caption{Diagrama de la luz incidente y luz reflectada}
    \vspace{0.5cm}
\end{figure}

Para que un BRDF se considere como PBR, ha de utilizar el modelo de microfacetas, ademas de cumplir con la ley de conservacion
de la energia, $\forall w_i \int_{\Omega} f(p, w_i, w_o) n\cdot{w_i} dw_o \leq 1$ y el principio de reciprocidad de Helmholtz,
el BRDF debe de ser simétrico, esto es que invertir la direccion de entrada y salida del BRDF no deberia afectar al resultado
$f(x, w_i, w_o) = f(x, w_o, w_i)$ sin embargo, los motores en tiempo real, frecuentemente incumplen el principio de reciprocidad,
no siendo fisicamente plausibles, pero sin generar artefactos.\\

    \subsection{Teor\'ia de microfacetas}

    \bgroup

        Aunque a escala macrosc\'opica podamos considerar una superficie como lisa, ninguna superficie es completamente lisa a nivel
        microsc\'opico. La teor\'ia de microfacetas utiliza una representaci\'on estad\'istica para modelar estas peque\~nas irregularidades.
        Para ello, la teor\'ia de microfacetas, considera la superficie de reflexi\'on como una superficie compuesta por una matriz de
        superficies mas peque\~nas que la longitud de onda de la luz pero muy peque\~nas desde el punto de vista de la c\'amara,
        completamente reflectantes, llamadas microfacetas, y cuyas diferentes orientaciones determinan la rugosidad de la superficie.
        Si las microfacetas est\'an completamente alineadas, las reflexiones ser\'an mas definidas, asimilandose a un espejo, mientras
        que las microfacetas apuntando en diferentes direcciones, daran como resultado una reflexion especular mas difusa.

        \begin{figure}[H]
            \vspace{0.5cm}
            \centering
            \frame{\includegraphics[scale=0.55]{macrosuperficie}}
            \caption{Superficie a escala microsc\'opica}
            \vspace{0.5cm}
        \end{figure}

    \egroup


    \subsection{Tipos de BRDFs}
    Existen diferentes tipos de BRDFs en funci\'on de las propiedades f\'isicas del material que se quiera representar. Los
    materiales met\'alicos utilizan un BRDF diferente a los diel\'ectricos, esto es as\'i porque los efectos de absorci\'on y
    dispersi\'on, son completamente diferentes en funci\'on de esta propiedad.\\

    En los materiales metalicos, la reflexi\'on especular es de color, la luz refractada es absorbida por el propio material
    y no existen efectos de dispersi\'on de la luz bajo la superficie, mientras que en los materiales diel\'ectricos, o no
    met\'alicos, el aspecto final del material esta determinado por una combinaci\'on de los fen\'omenos de absorci\'on y
    dispersi\'on de la luz.\\

    Por otra parte, podemos distinguir entre materiales anisotr\'opicos, como el metal pulido o el pelo, cuyas propiedades
    de reflexi\'on cambian al rotar sobre la superficie alrededor de su vector normal. Aunque casi todos los materiales son, en cierta medida, anisotr\'opicos, la
    aportaci\'on de esta propiedad sobre la apariencia final suele ser tan baja que se desprecia y se utilizan modelos
    isotr\'opicos.\\

    Adem\'as, seg\'un su origen, podemos distinguir principalmente entre tres tipos de modelos: emp\'iricos, te\'oricos y
    experimentales. Los modelos emp\'iricos no est\'an basados en f\'isica, si no que son simplificaciones de menor coste que
    aproximan un determinado fen\'omeno, ejemplos de modelos emp\'iricos son los modelos de Phong \autocite{phong} o el
    de Blinn-Phong \autocite{blinnphong}. Los modelos te\'oricos, como los basados en microfacetas, se basan en observaciones sobre el comportamiento de la luz y su
    interacci\'on sobre las superficies para crear un sistema de ecuaciones que describan este comportamiento sobre un modelo
    matem\'atico. Por \'ultimo, los modelo experimentales, se basan en datos de estudios experimentales con la luz a partir de los
    que crean funciones que se ajustan a los datos obtenidos en dichos estudios, el BRDF de Schlick es un ejemplo de modelo
    experimental.
    % Los diferentes tipos de BRDFs permiten para modelar diferentes propiedades de los materiales, isotropia o anisotropia,
    % transmitancia, reflexiones internas, etc. Podemos clasificar los diferentes tipos de BRDFs entre modelos analiticos y BDRFs
    % de datos adquiridos. Los modelos analiticos son funciones matemáticas que modelan diferentes efectos de la luz en funcion de
    % sus datos de entrada, mientras que los BRDFs de datos adquiridos, capturan el BRDF de un material con un gonioreflectometro,
    % y permiten una representacion muy precisa del material escaneado.

    % Comunmente, en la industria se utilizan modelos anal\'iticos, debido a su flexibilidad y rendimiento y el mas utilizado a
    % d\'ia de hoy en la industria, aunque con algunas variaciones en sus t\'erminos, sigue siendo el modelo de Cook-Torrance\autocite{cooktorrance}.
    % \todo[inline]{
    %     Metal. The appearance of the metal mainly depends on the direct reflection of light at the interface of the two media (ie, specular reflection). The metallic specular reflection color is a three-channel color, and R, G, and B are different. The light refracted into the metal is almost immediately absorbed by free electrons, and there is no scattering of light refracted into the metal.
    %     Non-Metal. Non-metal is a dielectric, and its overall appearance is mainly determined by its combination of absorption and scattering characteristics. Similarly, the interaction between non-metal and light is divided into reflection and refraction. According to the scattering and absorption characteristics of the medium type, refraction is divided into multiple categories:
    % }
    % \todo[inline]{
    %     The term isotropic is used to describe BRDFs that represent reflectance properties that
    %     are invariant with respect to rotation of the surface around the surface normal vector.
    %     Consider a small relatively smooth surface element and fix the light and viewer positions.
    %     Anisotropy, on the other hand, refers to BRDFs that describe reflectance properties that
    %     do exhibit change with respect to rotation of the surface around the surface normal
    %     vector. Some examples of materials that have anisotropic BRDFs are brushed metal,
    %     satin, and hair.
    % }

    \begin{figure}[H]
        \centering
        \frame{\includegraphics[scale=0.45]{schema_brdfs}}
        \caption{Esquema de diferentes modelos de BRDF seg\'un su origen}
        \vspace{0.5cm}
    \end{figure}

    \section{Limitaciones de los motores de renderizado en tiempo real}
    Los motores de renderizado en tiempo real deben utilizar algoritmos lo menos costosos posibles para garantizar un buen
    tiempo de respuesta. Es por ello que sus BRDFs, pueden ser m\'as sencillos que los implementados en sistemas de \textit{path-tracing},
    ofreciendo una soluci\'on de compromiso entre el rendimiento y los fen\'omenos f\'isicos representados por el modelo.\\
    
    Por otra parte, resolver la parte integral de la ecuaci\'on de render:

    \begin{center}$L_o(p, w_o) = \int_{\Omega} f_r(p, w_i, w_o)L_i(p, w_i)n\cdot{w_i}dw_i$\end{center}
    \singlespacing
    requiere lanzar rayos no en una direcci\'n si no desde todas las posibles direcciones sobre la semiesferea $\Omega$, por
    lo que no es posible en tiempo real, sin embargo, m\'as adelante analizaremos t\'ecnicas que p\'ermiten simular \'este
    efecto utilizando c\'alculos precomputados o aproximando la soluci\'on de forma anal\'itica.\\


    \section{BRDF Cook-Torrance}
    Es un modelo anal\'itico permite representar con gran realismo materiales electricos y dielectricos con diferentes
    grados de rugosidad y, en general, funciona bien para modelar cambios en el color que dependen del punto de vista.\\
    Este modelo permite estudiar por separado las dos componentes de la luz, especular y difuso:
    
    \begin{eqfloat}[!htb]
        \begin{equation}
            f_r = k_{d}f_{lambert} + k_sf_{cook-torrance}
        \end{equation}
        \caption{BRDF como suma de la componente difusa y especular}
    \end{eqfloat}
    
    siendo $k_d$ y $k_s$, parametros de peso que cumplen $k_d + k_s = 1$
    \singlespacing
    
    T\'ipicamente la componente difusa, utiliza el modelo de Lambert, que asume una distribuci\'on completamente uniforme a lo
    largo de la superficie:\\
    
    \begin{equation}
    f_{Lambert} = \frac{diffuse}{\pi}
    \end{equation}
    \singlespacing
    
    La componente especular es una funci\'on compuesta de otras tres funciones y un factor de normalizaci\'on en el
    denominador.\\
    
    \begin{equation}
        f(l, v) = \frac{F(w_i, h) G(w_i, h, w_o) D(h)} {4(n\cdot{w_i}) (n \cdot{w_o})}
    \end{equation}
    \singlespacing
    
        \subsection{T\'erminos del BRDF especular}
            El BRDF de Cook-Torrance est\'a compuesto por otras tres funciones y un factor de normalizaci\'on en el denominador.
            Las funciones D, F y G, se corresponden con la funci\'on de distribuci\'on de las normales, la ecuaci\'on de Fresnel,
            y la funci\'on de geometr\'ia.\\
    
            $D(l, v, h)$  es la funci\'on de distribuci\'on de las normales y se encarga de representar la rugosidad de una superficie
            y es el t\'ermino que afecta en mayor grado a la forma y taman\~no del brillo especular. T\'ipicamente el par\'ametro
            \textit{roughness} se utiliza para representar la cantidad de microfacetas alineadas con la normal h, cuanta mayor sea la
            la cantidad de de microfacetas alineadas, m\'as lisa parecer\'a la superficie.
    
            $$
            D(h) = \frac{\alpha^2}{\pi((n\cdot{h})^2(\alpha^2 - 1) + 1)^2}
            $$
    
            \begin{figure}[H]
                \vspace{0.5cm}
                \centering
                \frame{\includegraphics[scale=0.65]{ndf}}
                \caption{Representaci\'on gr\'afica de la funci\'on de geometr\'ia}
            \end{figure}
    
            $G(w_o)$ es la funci\'on de geometr\'ia. Esta funci\'on tiene en cuenta las oclusiones debidas a la propia superficie.
            Mientras que el t\'ermino de distribucion de las normales, $D(h)$, define la concentraci\'on de microfacetas con una normal h,
            no define si esas microfacetas son visible o no. El t\'ermino de geometr\'ia tiene en cuenta las dos cosas, la sombra
            y el enmascaramiento. La sombra representa que una microfaceta no es visible desde la direcci\'on de la luz, mientras que el
            enmascaramiento significa que una microfaceta no es visible desde la direcci\'on de vista y, por tanto, no contribuyen
            a la reflexi\'on.
    
            $$
            G(w_i, w_o) = G(w_i)G(w_o)
            $$
            \singlespacing
            $$
            G(w_o) = \frac{n\cdot{w_o}}{(n\cdot) (1 - k) + k}
            $$
            \singlespacing
            $$
            G(w_i) = \frac{n\cdot{w_i}}{(n\cdot) (1 - k) + k}
            $$
            \singlespacing
    
            \begin{equation}
            k = \frac{(roughness + 1)^2}{8}
            \end{equation}
    
            \begin{figure}[H]
                \vspace{0.5cm}
                \centering
                \frame{\includegraphics[scale=0.65]{microgeometria}}
                \caption{Representaci\'on gr\'afica de la sombra y el enmascaramiento en la funci\'on de geometr\'ia}
            \end{figure}
    
            $F(l, v, h)$ es la funci\'on de Fresnel. Representa la radiancia reflejada por un material seg\'un el \'angulo de incidencia
            de la luz. Para la mayor\'ia de los materiales no met\'alicos, las reflexiones son m\'as intensas bajo cuando el \'angulo
            de incidencia es muy agudo. En la mayor\'ia de los algoritmos de sombreado, el fresnel se utiliza a nivel de la macrosuperficie,
            sin embargo, al aplicar el fresnel sobre la normal de la microfacetas, se consigue un mayor grado de realismo para los valores
            altos de \textit{roughness}.
    
            $$
            F_{Schlick}(F_o, w_i, h) = F_o + (1 - F_o) (1 - (l\cdot{h}))^5
            $$
    
            \begin{figure}[H]
                \vspace{0.5cm}
                \centering
                \frame{\includegraphics[scale=0.65]{fresnel}}
                \caption{Representaci\'on gr\'afica de la funci\'on de geometr\'ia}
            \end{figure}
            \singlespacing
    
            El denominador es una caracter\'istica t\'ipica de los modelos de microfacetas y es un factor de correci\'on debido al cambio
            de espacio entre el espacio local de las microfacetas y el espacio local de la macrosuperficie. Para modelos que no
            incluyen este factor, se puede aplicar un factor de sombreado directamente multiplicando el t\'ermino de geometr\'ia por
    
            \begin{equation}
            \frac{1}{4(n\cdot{l}) (n\cdot{v})}
            \end{equation}
            \singlespacing
    
            % La funci\on de distribuci\'on de las normales, aproxima la cantidad de microfacetas aproxima la alineadas con el
            % vector h, el medio entre el rayo de luz y el punto del vista. El par\'ametro que modela la rugosidad de una superficie
            % afecta en gran medida al resultado. Este es el factor m\'as caracter\'istico del modelo de microfacetas.
            % La funci\'on de geometr\'ia describe la cantidad de rayos ocluidos por las propias microfacetas.\\
    
            % Finalmente, la ecuaci\'on de fresnel describe el ratio de reflexi\'on de una superficie bajo diferentes \'angulos de
            % incidencia.\\
    
    \section{Disney Principled BRDF}
    \todo[inline]{Modelo emp\'irico basado en las observaciones de la tabla MERL}
    
    \begin{figure}[H]
        \vspace{0.5cm}
        \centering
        \frame{\includegraphics[scale=0.47]{slicespace}}
        \caption{Muestras de materiales MERL, junto a un esquema explicativo de las im\'agenes muestreadas}
        \vspace{0.5cm}
    \end{figure}
    
    El BRDF de Disney fue presentado por Brent Burley en 2012 y es el utilizado por Disney en sus peliculas de animacion. Es,
    junto al modelo de Cook-Torrance, el mas utilizado en los motores de tiempo real, sin embargo, el modelo de Burley, es mas
    amigable para los artistas, a costa de no ser completamente basado en fisica. Los parametros tienen nombres y valores que
    definen que el aspecto de los materiales son mas intuitivos. Los lobulos adicionales, a diferencia del modelo de Cook-Torrance,
    pueden variar mucho en forma y tamanho, por lo que es un modelo muy flexible, que permite representar con gran realismo una
    amplia variedad de materiales.\\
    
    Es conocido como \textit{principled} por cumplir una serie de m\'aximas, que se respetan en el modelo por encima de las leyes fisicas.
    El modelo ha de ser intuitivo y no utilizar parametros que se refieran al modelo basado en fisica, debe tener el menor
    numero de parametros posible, deben de estar normalizados, algunos valores pueden exceder su rango para permitir mayor
    expresividad y, finalmente, todas las combinaciones de parametros deben de ser robustas y plausibles. Para ello utiliza los
    par\'ametros: \textit{baseColor}, \textit{subsurface}, \textit{metal}, \textit{specular}, \textit{specularTint}, \textit{roughness},
    \textit{sheen}, \textit{sheenTint}, \textit{clearCoat}, \textit{clearcoatGloss}
    
    \begin{figure}[H]
        \vspace{0.5cm}
        \centering
          \frame{\includegraphics[scale=0.47]{disney}}
        \caption{Modelo de Disney}
    \end{figure}
    % \begin{itemize}
    % 	\item \textit{baseColor}: el color de la superficie, comunmente utiliza un mapa.
    % 	\item \textit{subsurface}: aproximacion para controlar el color de la reflexion difusa.
    % 	\item \textit{metal}: interpolaci\'on lineal entre los dos modelos. El met\'alico no tiene
    %     componente de reflexi\'on difusa y su componente de reflexi\'on especular es del mismo color que el color base. 
    %     \item \textit{specular}: cantidad de luz incidente reflejada, se utiliza para controlar de forma intuitiva el \'indice de refracci\'on de
    %     una superficie.
    % 	\item \textit{specularTint}: par\'ametro que permite a los artistas controlar el color de la reflexi\'on especular sobre el color
    % 	base.
    % 	\item \textit{roughness}: describe la rugosidad de una superficie, afectando a la reflexi\'on difusa y especular.
    % 	\item \textit{sheen}: permite mayor grado de control sobre la reflexi\'on especular, muy \'util sobre tejidos.
    % 	\item \textit{sheenTint}: color del \textit{sheen}
    % 	\item \textit{clearCoat}: un l\'obulo extra.
    % 	\item \textit{clearcoatGloss}: permite controlar la rugosidad de este l\'obulo.
    % \end{itemize}
        
        \subsection{Componente difusa}
    
        En base a las observaciones sobre los datos de MERL, el modelo difuso parece muy oscuro hacia los bordes, y aplicando
        la funci\'on de Fresnel, para intentar conseguir un modelo f\'isicamente plausible, parece oscurecer m\'as los bordes,
        acentuando el problema. Disney desarroll\'o un modelo emp\'irico, que utiliza la aproximaci\'on de Fresnel Schlick:
    
        \begin{equation}
            (1 - F(\theta_l) (1 - F(\theta_d)))
        \end{equation}
        \singlespacing
    
    
        Modific\'andolo para conseguir que la retrodispersi\'on dependa del valor de \textit{roughness} y as\'i
        adaptarse mejor a los datos obtenidos de MERL.
    
        \begin{equation}
        f_d = \frac{baseColor}{\pi}
        \left(  1 + (F_{D90} - 1)(1 - cos\theta_{wi})^5  \right)
        \left(  1 + (F_{D90} - 1)(1 - cos\theta_{wo})^5  \right)
        \end{equation}
        
        $$
        F_{D90} = 0.5 + 2roughness\cdot{cos^2\theta_d}
        $$
    
        \subsection{Componente especular}
            \subsubsection{F, t\'ermino de Fresnel}
                Para el especular, la aproximaci\'on de Fresnel Schlick es lo suficientemente precisa y mucho menos costosa que
                que la ecuaci\'on completa de Fresnel.
    
                \begin{equation}
                (1 - F(\theta_l) (1 - F(\theta_d)))^5
                \end{equation}
    
                $F_0$ representa la reflectancia de una incidencia del mismo \'angulo que la normal. $\theta_d$ es el \'angulo
                entre el vector $h$, y el de vista $w_o$. Es acrom\'actico para los diel\'ectricos y crom\'atico para los metales.
    
            \subsubsection{G, t\'ermino de geometr\'ia}
                En el t\'ermino G, Disney utiliza dos modelos diferentes, uno para el l\'obulo primario y otro para el de clearcoat.
                Para el l\'obulo primario, utiliza el modelo de Smith GGX, remapeando el valor de \textit{roughness} para evitar
                ganar demasiada energ\'ia hacia los bordes de materiales brillantes.
    
                $$
                G(l, v, h) = G_{GGX}(l)G_{GGX}(v)
                $$
    
                $$
                G_{GGX}(v) = \frac
                {2 (n \cdot{v})}
                {(n \cdot{v}) + \sqrt{ \alpha^2 + (1 - \alpha)^2 (n \cdot{v})^2 }}
                $$
    
                \begin{equation}
                \alpha = (0.5 + roughness / 2)^2
                \end{equation}
                \singlespacing
    
                Para el l\'obulo secundario se utiliza un valor fijo de $\alpha = 0.25$.
    
            \subsubsection{D, t\'ermino de distribuci\'on de las normales}
                La distribuci\'on GGX es equivalente a la de Trowbridge-Reitz, no tiene una cola lo suficientemente larga para la
                mayor\'ia de materiales.
    
                \begin{equation}
                    D_{TR} = \frac
                    {c}
                    {(\alpha^2 cos^2 \theta_h + sin^2 \theta_h)^2}
                \end{equation}
                \singlespacing
    
                El modelo de Disney utiliza un exponente en el denominador que permite mayor control sobre el radio de la reflexi\'on
                especular, llamando a \'este t\'ermino Generalized Trowbridge-Reitz, o GTR:
    
                \begin{equation}
                    D_{GTR} = \frac
                    {c}
                    {(\alpha^2 cos^2 \theta_h + sin^2 \theta_h)^\gamma}
                \end{equation}
                \singlespacing
    
                Los dos l\'obulos, primario y secundario utilizan la distribuci\'on GTR. El l\'obulo primario,
                que representa la reflexi\'on del material base, utiliza $\gamma = 2$ y puede ser diel\'etrico o met\'alico,
                isotr\'opico o anisotr\'opico. Por otra parte, el l\'obulo secundario representa la capa de \textit{clearcoat}
                sobre el material base, utiliza $\gamma = 1$ y suele ser isotr\'opico y no met\'alico.
    
    
    
    \section{Iluminaci\'on indirecta en tiempo real}
    
    Para los efectos de iluminacion global, se necesita conocer la irradiancia proviniente en todas
    direcciones $w_i$ sobre la esfera $\Omega$.\\
    
    \begin{figure}[H]
        \vspace{0.5cm}
        \centering
          \frame{\includegraphics[scale=0.5]{hemisphere}}
        \caption{hemisphere}
      \end{figure}
      \singlespacing
    
    La ecuaci\'on de render describe la radiancia de salida sobre un punto.
    \begin{equation}
    L_o(p, w_o) = \int_{\Omega} f_r(p, w_i, w_o)L_i(p, w_i)n\cdot{w_i}dw_i
    \end{equation}
    \singlespacing
    
    Siendo $\Omega$ la hemiesfera centrada en el punto sobre la que calculamos la irradiancia,
    $f_r$ representa el BRDF, $Li$, la irradiancia de la escena, mientras que $n\cdot{w_i}$ toma en
    cuenta el \'angulo entre la de incidencia de la luz sobre la superficie.
    
    \begin{equation}
    L_o(p, w_o) = \int_{\Omega} (k_d + \frac{c}{\pi} + 
    k_s \frac{DFG}{4(w_o\cdot{n})(w_i\cdot{n})})L_i(p, w_i)n\cdot{w_i}dw_i
    \end{equation}
    \singlespacing
    
    Los terminos $k_s$ y $k_d$ de la ecuacion de reflectancia son independientes, por lo que, de la
    misma forma que en la iluminacion directa, los componentes se pueden separar en difuso y
    especular.
    
    \begin{equation}
    L_o(p, w_o) = \int_{\Omega}
    (k_d \frac{c}{\pi}) L_i(p, w_i)n\cdot{w_i}dw_i +
    \int_{\Omega} 
    k_s \frac{DFG}{4(w_o\cdot{n})(w_i\cdot{n})})L_i(p, w_i)n\cdot{w_i}dw_i
    \end{equation}
    \singlespacing
    
        \subsection{Componente difusa}
        La soluci\'on de la integral de la irradiancia de salida sobre $\Omega$ requiere samplear el entorno
        en todas las direcciones posibles. Es por ello que en tiempo real, la soluci\'on consiste en
        precomputar este c\'alculo.\\
    
        Habiendo separado la ecuaci\'on para la componente difusa y especular, podemos observar que el t\'ermino del difuso
        de Lambert es constante y lo podemos sacar de la integral.
    
        \begin{equation}
        L_o(p, w_o) = \int_{\Omega}
        (k_d \frac{c}{\pi}) L_i(p, w_i)n\cdot{w_i}dw_i=
        k_d \frac{c}{\pi} \int_{\Omega}
        (L_i(p, w_i)n\cdot{w_i}dw_i
        \end{equation}
        \singlespacing
    
            \subsubsection{Mapas de irradiancia}
            T\'ecnica IBL (\textit{Image Based Lighting}), que permite precalcular la irradiacia del entorno, utilizando una im\'agen
            de referencia y almacenarla en una nueva textura, el mapa de irradiancia. Para ello se 
            \todo[inline]{
                Convolution is applying some computation to each entry in a data set considering all other entries in the data set;
                the data set being the scene's radiance or environment map. Thus for every sample direction in the cubemap, we take
                all other sample directions over the hemisphere $\Omega$ into account.
                To convolute an environment map we solve the integral for each output wo sample direction by discretely sampling a
                large number of directions wi over the hemisphere $\Omega$ and averaging their radiance. The hemisphere we build the sample
                directions wi from is oriented towards the output wo sample direction we're convoluting.
            }
            \todo[inline]{
                computed sum of all indirect diffuse light of the scene hitting some surface aligned along direction wo. Such a
                cubemap is known as an irradiance map seeing as the convoluted cubemap effectively allows us to directly sample the
                scene's (pre-computed) irradiance from any direction wo.
            }
    
            \begin{figure}[H]
                \vspace{0.5cm}
                \centering
                \frame{\includegraphics[scale=0.5]{irradiance_map}}
                \caption{Mapa de irradiancia como \textit{cubemap}}
            \end{figure}
            \singlespacing
    
            \subsubsection{Esf\'ericos harm\'onicos}
            \todo[inline]{
                \url{https://developer.nvidia.com/gpugems/gpugems2/part-ii-shading-lighting-and-shadows/chapter-10-real-time-computation-dynamic}
            }
        \subsection{Componente especular}
    
    



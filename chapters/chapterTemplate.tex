\chapter{My first chapter}
Especificadas las características y requerimientos de la RA se deducen que las experiencias del usuario quedan reducida a la interacción de éste con la pantalla del dispositivo que se esté empleando. Así mismo, la capacidad de crear escenarios complejos compuestos por elementos virtuales también es una de sus limitaciones en cuanto a la dependencia del uso de marcadores o activadores de RA. La técnica de realidad mixta presenta una solución satisfactoria a estas dos cuestiones, pudiendo ofrecer una experiencia menos limitada interactivamente para el usuario, aunque también presenta otras desventajas.
La Realidad Mixta se ha considerado la combinación entre la realidad aumentada, virtualidad aumentada y realidad virtual, que permite crear un espacio en el que admite la interacción del usuario tanto con objetos reales como virtuales, es decir, objetos generados sintéticamente de apariencia real. Esta tecnología tiene la capacidad de aunar la interactividad de la realidad física y el poder visual de la realidad aumentada, por lo que se plantea un paso por delante a la realidad aumentada en capacidad de inmersión. Sin embargo, la tecnología también cuenta con sus limitaciones, como es que el uso de esta tecnología está condicionado por los distintos cascos HMD disponibles en el mercado, lo cual obliga a los desarrolladores a adaptar las soluciones a las posibilidades y los requisitos que marca cada gafa de realidad aumentada/mixta. Tampoco pasa desapercibido que la distribución de esta tecnología se realiza en formato gafa y, por tanto, el coste del dispositivo impacta directamente en el alcance que tiene frente a la realidad aumentada soportada por smartphones. Debido a que esta tecnología depende de la última innovación en gafas (Microsoft y Magic Leap, máxime) lo que hace que su actual aplicación sea principalmente en el mercado B2B, siendo el gran reto de los próximos años conseguir dispositivos más accesibles para el desarrollo del mercado B2C, que permita democratizar el uso de AR/MR en la sociedad actual.
Por otro lado, el proceso que realiza esta gafa no dista en exceso del funcionamiento de uso de la realidad aumentada. Su principal diferencia consiste en que la gafa compone un mapa 3D del entorno mediante técnicas SLAM y posteriormente, analiza este escaneo de las superficies detectadas para habilitar la colocación de elementos visuales en los lugares exactos. Esta representación de los elementos virtuales se proyecta en las pantallas transparentes del casco, lo que nos permite ver a través de ellas el mundo real. Actualmente, la mayor parte de las interacciones con los objetos virtuales dependen del uso de los controladores propios del headset, aunque cada vez más se permite el uso de las manos, integrando el reconocimiento de gestos simples.
La RM sigue siendo una tecnología en estado embrionario con gran potencial de innovación futura, la cual todavía requiere de muchas horas de desarrollo y trabajo por parte de los profesionales del sector para generar contenidos espectaculares que demuestren la potencia de la herramienta. No obstante, puede resultar una gran aliada para lograr escenografías más inversivas tanto para el propósito educativo como formativo.
Conviene dejar constancia que la fundamentación del presente trabajo asume todos los beneficios destacados en el uso de la realidad aumentada en enseñanza igualmente aplicables a la realidad mixta, pues la RM aportará como extra una mayor capacidad de interacción e inmersión, pero sus propósitos y método de uso en enseñanza son idénticos para ambas técnicas. Finalmente, se listan los principios que debe satisfacer la incorporación de la RA en entornos de enseñanza:
Especificadas las características y requerimientos de la RA se deducen que las experiencias del usuario quedan reducida a la interacción de éste con la pantalla del dispositivo que se esté empleando. Así mismo, la capacidad de crear escenarios complejos compuestos por elementos virtuales también es una de sus limitaciones en cuanto a la dependencia del uso de marcadores o activadores de RA. La técnica de realidad mixta presenta una solución satisfactoria a estas dos cuestiones, pudiendo ofrecer una experiencia menos limitada interactivamente para el usuario, aunque también presenta otras desventajas.
La Realidad Mixta se ha considerado la combinación entre la realidad aumentada, virtualidad aumentada y realidad virtual, que permite crear un espacio en el que admite la interacción del usuario tanto con objetos reales como virtuales, es decir, objetos generados sintéticamente de apariencia real. Esta tecnología tiene la capacidad de aunar la interactividad de la realidad física y el poder visual de la realidad aumentada, por lo que se plantea un paso por delante a la realidad aumentada en capacidad de inmersión. Sin embargo, la tecnología también cuenta con sus limitaciones, como es que el uso de esta tecnología está condicionado por los distintos cascos HMD disponibles en el mercado, lo cual obliga a los desarrolladores a adaptar las soluciones a las posibilidades y los requisitos que marca cada gafa de realidad aumentada/mixta. Tampoco pasa desapercibido que la distribución de esta tecnología se realiza en formato gafa y, por tanto, el coste del dispositivo impacta directamente en el alcance que tiene frente a la realidad aumentada soportada por smartphones. Debido a que esta tecnología depende de la última innovación en gafas (Microsoft y Magic Leap, máxime) lo que hace que su actual aplicación sea principalmente en el mercado B2B, siendo el gran reto de los próximos años conseguir dispositivos más accesibles para el desarrollo del mercado B2C, que permita democratizar el uso de AR/MR en la sociedad actual.
Por otro lado, el proceso que realiza esta gafa no dista en exceso del funcionamiento de uso de la realidad aumentada. Su principal diferencia consiste en que la gafa compone un mapa 3D del entorno mediante técnicas SLAM y posteriormente, analiza este escaneo de las superficies detectadas para habilitar la colocación de elementos visuales en los lugares exactos. Esta representación de los elementos virtuales se proyecta en las pantallas transparentes del casco, lo que nos permite ver a través de ellas el mundo real. Actualmente, la mayor parte de las interacciones con los objetos virtuales dependen del uso de los controladores propios del headset, aunque cada vez más se permite el uso de las manos, integrando el reconocimiento de gestos simples.
La RM sigue siendo una tecnología en estado embrionario con gran potencial de innovación futura, la cual todavía requiere de muchas horas de desarrollo y trabajo por parte de los profesionales del sector para generar contenidos espectaculares que demuestren la potencia de la herramienta. No obstante, puede resultar una gran aliada para lograr escenografías más inversivas tanto para el propósito educativo como formativo.
Conviene dejar constancia que la fundamentación del presente trabajo asume todos los beneficios destacados en el uso de la realidad aumentada en enseñanza igualmente aplicables a la realidad mixta, pues la RM aportará como extra una mayor capacidad de interacción e inmersión, pero sus propósitos y método de uso en enseñanza son idénticos para ambas técnicas. Finalmente, se listan los principios que debe satisfacer la incorporación de la RA en entornos de enseñanza:
Random citation \autocite[1]{DUMMY1} embeddeed in text.



\begin{lstlisting}[caption=My Javascript Example]
Name.prototype = {
  methodName: function(params){
    var doubleQuoteString = "some text";
    var singleQuoteString = 'some more text';
    // this is a comment
    if(this.confirmed != null && typeof(this.confirmed) == Boolean && this.confirmed == true){
      document.createElement('h3');
      $('#system').append("This looks great");
      return false;
    } else {
      throw new Error;
    }
  }
}
\end{lstlisting}

\begin{lstlisting}[caption=My Javascript Example]
  #include <stdio.h>
  int main() {
      double a, b, product;
      printf("Enter two numbers: ");
      scanf("%lf %lf", &a, &b);  
    
      // Calculating product
      product = a * b;
  
      // Result up to 2 decimal point is displayed using %.2lf
      printf("Product = %.2lf", product);
      
      return 0;
  }
\end{lstlisting}

\section{My first section}
Especificadas las características y requerimientos de la RA se deducen que las experiencias del usuario quedan reducida a la interacción de éste con la pantalla del dispositivo que se esté empleando. Así mismo, la capacidad de crear escenarios complejos compuestos por elementos virtuales también es una de sus limitaciones en cuanto a la dependencia del uso de marcadores o activadores de RA. La técnica de realidad mixta presenta una solución satisfactoria a estas dos cuestiones, pudiendo ofrecer una experiencia menos limitada interactivamente para el usuario, aunque también presenta otras desventajas.
La Realidad Mixta se ha considerado la combinación entre la realidad aumentada, virtualidad aumentada y realidad virtual, que permite crear un espacio en el que admite la interacción del usuario tanto con objetos reales como virtuales, es decir, objetos generados sintéticamente de apariencia real. Esta tecnología tiene la capacidad de aunar la interactividad de la realidad física y el poder visual de la realidad aumentada, por lo que se plantea un paso por delante a la realidad aumentada en capacidad de inmersión. Sin embargo, la tecnología también cuenta con sus limitaciones, como es que el uso de esta tecnología está condicionado por los distintos cascos HMD disponibles en el mercado, lo cual obliga a los desarrolladores a adaptar las soluciones a las posibilidades y los requisitos que marca cada gafa de realidad aumentada/mixta. Tampoco pasa desapercibido que la distribución de esta tecnología se realiza en formato gafa y, por tanto, el coste del dispositivo impacta directamente en el alcance que tiene frente a la realidad aumentada soportada por smartphones. Debido a que esta tecnología depende de la última innovación en gafas (Microsoft y Magic Leap, máxime) lo que hace que su actual aplicación sea principalmente en el mercado B2B, siendo el gran reto de los próximos años conseguir dispositivos más accesibles para el desarrollo del mercado B2C, que permita democratizar el uso de AR/MR en la sociedad actual.
Por otro lado, el proceso que realiza esta gafa no dista en exceso del funcionamiento de uso de la realidad aumentada. Su principal diferencia consiste en que la gafa compone un mapa 3D del entorno mediante técnicas SLAM y posteriormente, analiza este escaneo de las superficies detectadas para habilitar la colocación de elementos visuales en los lugares exactos. Esta representación de los elementos virtuales se proyecta en las pantallas transparentes del casco, lo que nos permite ver a través de ellas el mundo real. Actualmente, la mayor parte de las interacciones con los objetos virtuales dependen del uso de los controladores propios del headset, aunque cada vez más se permite el uso de las manos, integrando el reconocimiento de gestos simples.
La RM sigue siendo una tecnología en estado embrionario con gran potencial de innovación futura, la cual todavía requiere de muchas horas de desarrollo y trabajo por parte de los profesionales del sector para generar contenidos espectaculares que demuestren la potencia de la herramienta. No obstante, puede resultar una gran aliada para lograr escenografías más inversivas tanto para el propósito educativo como formativo.
Conviene dejar constancia que la fundamentación del presente trabajo asume todos los beneficios destacados en el uso de la realidad aumentada en enseñanza igualmente aplicables a la realidad mixta, pues la RM aportará como extra una mayor capacidad de interacción e inmersión, pero sus propósitos y método de uso en enseñanza son idénticos para ambas técnicas. Finalmente, se listan los principios que debe satisfacer la incorporación de la RA en entornos de enseñanza:
Especificadas las características y requerimientos de la RA se deducen que las experiencias del usuario quedan reducida a la interacción de éste con la pantalla del dispositivo que se esté empleando. Así mismo, la capacidad de crear escenarios complejos compuestos por elementos virtuales también es una de sus limitaciones en cuanto a la dependencia del uso de marcadores o activadores de RA. La técnica de realidad mixta presenta una solución satisfactoria a estas dos cuestiones, pudiendo ofrecer una experiencia menos limitada interactivamente para el usuario, aunque también presenta otras desventajas.
La Realidad Mixta se ha considerado la combinación entre la realidad aumentada, virtualidad aumentada y realidad virtual, que permite crear un espacio en el que admite la interacción del usuario tanto con objetos reales como virtuales, es decir, objetos generados sintéticamente de apariencia real. Esta tecnología tiene la capacidad de aunar la interactividad de la realidad física y el poder visual de la realidad aumentada, por lo que se plantea un paso por delante a la realidad aumentada en capacidad de inmersión. Sin embargo, la tecnología también cuenta con sus limitaciones, como es que el uso de esta tecnología está condicionado por los distintos cascos HMD disponibles en el mercado, lo cual obliga a los desarrolladores a adaptar las soluciones a las posibilidades y los requisitos que marca cada gafa de realidad aumentada/mixta. Tampoco pasa desapercibido que la distribución de esta tecnología se realiza en formato gafa y, por tanto, el coste del dispositivo impacta directamente en el alcance que tiene frente a la realidad aumentada soportada por smartphones. Debido a que esta tecnología depende de la última innovación en gafas (Microsoft y Magic Leap, máxime) lo que hace que su actual aplicación sea principalmente en el mercado B2B, siendo el gran reto de los próximos años conseguir dispositivos más accesibles para el desarrollo del mercado B2C, que permita democratizar el uso de AR/MR en la sociedad actual.
Por otro lado, el proceso que realiza esta gafa no dista en exceso del funcionamiento de uso de la realidad aumentada. Su principal diferencia consiste en que la gafa compone un mapa 3D del entorno mediante técnicas SLAM y posteriormente, analiza este escaneo de las superficies detectadas para habilitar la colocación de elementos visuales en los lugares exactos. Esta representación de los elementos virtuales se proyecta en las pantallas transparentes del casco, lo que nos permite ver a través de ellas el mundo real. Actualmente, la mayor parte de las interacciones con los objetos virtuales dependen del uso de los controladores propios del headset, aunque cada vez más se permite el uso de las manos, integrando el reconocimiento de gestos simples.
La RM sigue siendo una tecnología en estado embrionario con gran potencial de innovación futura, la cual todavía requiere de muchas horas de desarrollo y trabajo por parte de los profesionales del sector para generar contenidos espectaculares que demuestren la potencia de la herramienta. No obstante, puede resultar una gran aliada para lograr escenografías más inversivas tanto para el propósito educativo como formativo.
Conviene dejar constancia que la fundamentación del presente trabajo asume todos los beneficios destacados en el uso de la realidad aumentada en enseñanza igualmente aplicables a la realidad mixta, pues la RM aportará como extra una mayor capacidad de interacción e inmersión, pero sus propósitos y método de uso en enseñanza son idénticos para ambas técnicas. Finalmente, se listan los principios que debe satisfacer la incorporación de la RA en entornos de enseñanza:

\subsection{My first subsection}
Especificadas las características y requerimientos de la RA se deducen que las experiencias del usuario quedan reducida a la interacción de éste con la pantalla del dispositivo que se esté empleando. Así mismo, la capacidad de crear escenarios complejos compuestos por elementos virtuales también es una de sus limitaciones en cuanto a la dependencia del uso de marcadores o activadores de RA. La técnica de realidad mixta presenta una solución satisfactoria a estas dos cuestiones, pudiendo ofrecer una experiencia menos limitada interactivamente para el usuario, aunque también presenta otras desventajas.
La Realidad Mixta se ha considerado la combinación entre la realidad aumentada, virtualidad aumentada y realidad virtual, que permite crear un espacio en el que admite la interacción del usuario tanto con objetos reales como virtuales, es decir, objetos generados sintéticamente de apariencia real. Esta tecnología tiene la capacidad de aunar la interactividad de la realidad física y el poder visual de la realidad aumentada, por lo que se plantea un paso por delante a la realidad aumentada en capacidad de inmersión. Sin embargo, la tecnología también cuenta con sus limitaciones, como es que el uso de esta tecnología está condicionado por los distintos cascos HMD disponibles en el mercado, lo cual obliga a los desarrolladores a adaptar las soluciones a las posibilidades y los requisitos que marca cada gafa de realidad aumentada/mixta. Tampoco pasa desapercibido que la distribución de esta tecnología se realiza en formato gafa y, por tanto, el coste del dispositivo impacta directamente en el alcance que tiene frente a la realidad aumentada soportada por smartphones. Debido a que esta tecnología depende de la última innovación en gafas (Microsoft y Magic Leap, máxime) lo que hace que su actual aplicación sea principalmente en el mercado B2B, siendo el gran reto de los próximos años conseguir dispositivos más accesibles para el desarrollo del mercado B2C, que permita democratizar el uso de AR/MR en la sociedad actual.
Por otro lado, el proceso que realiza esta gafa no dista en exceso del funcionamiento de uso de la realidad aumentada. Su principal diferencia consiste en que la gafa compone un mapa 3D del entorno mediante técnicas SLAM y posteriormente, analiza este escaneo de las superficies detectadas para habilitar la colocación de elementos visuales en los lugares exactos. Esta representación de los elementos virtuales se proyecta en las pantallas transparentes del casco, lo que nos permite ver a través de ellas el mundo real. Actualmente, la mayor parte de las interacciones con los objetos virtuales dependen del uso de los controladores propios del headset, aunque cada vez más se permite el uso de las manos, integrando el reconocimiento de gestos simples.
La RM sigue siendo una tecnología en estado embrionario con gran potencial de innovación futura, la cual todavía requiere de muchas horas de desarrollo y trabajo por parte de los profesionales del sector para generar contenidos espectaculares que demuestren la potencia de la herramienta. No obstante, puede resultar una gran aliada para lograr escenografías más inversivas tanto para el propósito educativo como formativo.
Conviene dejar constancia que la fundamentación del presente trabajo asume todos los beneficios destacados en el uso de la realidad aumentada en enseñanza igualmente aplicables a la realidad mixta, pues la RM aportará como extra una mayor capacidad de interacción e inmersión, pero sus propósitos y método de uso en enseñanza son idénticos para ambas técnicas. Finalmente, se listan los principios que debe satisfacer la incorporación de la RA en entornos de enseñanza:

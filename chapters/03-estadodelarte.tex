\chapter{Estado del arte}

En 1970 se presenta el primer algoritmo de sombreado en gr\'aficos rasterizados \autocite{gouraud}, posteriormente 
Phong \autocite{phong} presentada un modelo para la componente especular, revisado y mejorado posteriormente en el modelo Blinn-Phong
\autocite{blinnphong} por James Blinn. Estos modelos consiguen describir el comportamiento del material en todos sus puntos,
estableciendo un ratio entre la luz que llega a una superficie y la luz reflejada, sin embargo no tienen en cuenta el modelo f\'isico de
la luz, por lo que no podemos hablar de materiales PBR hasta 1981, a\~no en el que Robert Cook y Kenneth Torrance presentan su modelo
de BRDF basado en microfacetas \autocite{cooktorrance}

\section{Componentes y t\'erminos del BRDF basado en microfacetas}
\todo[inline]{
    Blinn \autocite{blinn77}, Beckmann \autocite{beckmann}, Walter \autocite{ggx}, Neumann \autocite{neumann}, Kelemen
    \autocite{kelemen}, Smith \autocite{smith}, Karis \autocite{karis}, \autocite{reed}
}

El modelo de Cook-Torrance \autocite{cooktorrance}, que explicaremos en detalle en el Cap\'itulo 4, sus principales aportaciones son
separar el BRDF como la suma de sus dos componentes, especular y difuso y utilizar un BRDF especular basado en microfacetas. El BRDF
especular es la como la combinaci\'on de los t\'erminos de distribuci\'on de las normales, el de geometr\'ia o visibilidad que modela
los fen\'omenos de sombra y enmascaramiento y el de Fresnel.
% El trabajo utiliza la funcion de distribuci\'on de las normales de Beckmann \autocite{beckmann}, mientras que para el t\'ermino de Fresenl presenta
% una aproximaci\'on original de \'este trabajo. Aunque con variaciones en su componente difusa o los t\'erminos del BRDF especular que
% repasaremos a continuaci\'on, este sigue siendo el modelo de referencia hoy en d\'ia para materiales PBR.

    \subsection{Componente difusa}
        Johann Heinrich Lambert introdujo en 1760 \autocite{lambert} el concepto de difusi\'on perfecta, superficies cuya radiancia no
        dependen del punto de vista, si no que parecen tener la misma radiancia desde todas las direcciones. Este modelo fue el utilizado
        por Cook-Torrance \autocite{cooktorrance} y se sigue utilizando con frecuencia a d\'ia de hoy, ya que supone una buena
        aproximaci\'on para una gran parte de materiales. Sin embargo, este modelo ideal no tiene en cuenta la conservaci\'on
        de la energ\'ia y cuando se utiliza en conjunto con un BRDF basado en f\'isica, podemos observar un oscurecimiento
        en las zonas de mayor rugosidad, que se acent\'ua a medida que aumenta al incrementar el \'angulo de incidencia de la luz.
        El modelo de Oren-Nayar \autocite{orennayar}, presenta por primera vez un t\'ermino difuso f\'isicamente plausible,
        basado en la teor\'ia de microfacetas, especialmente en el trabajo de Torrance-Sparrow \autocite{torrancesparrow}. M\'as
        tarde, los modelos de Disney \autocite{disney12} \autocite{disney15}, Yoshiharu Gotanda \autocite{gotanda14} o Eal Hammon
        \autocite{earlhammon} presentaron sucesivas aproximaciones a la componente difusa basada en f\'isica teniendo en cuenta la
        distribuci\'on de las microfacetas dependiendo de los t\'erminos utilizados para la componente especular.

        % en computaci\'n gr\'afica para representar la componente difusa de un material,

        % Superficies con una reflexi\'on ideal, concepto que fue introducido en 1760 por Johann Heinrich Lambert en su libro fotometr\'ia,
        % este tipo de superficies se conocen como perfectamente difusas o superficies Lambertianas en su honor.
        % % Lambertian surfaces are ideally diusing surfaces, whose radiance is in dependent of viewing direction
        % % they appear equally brightfrom all directions.
        % The apparent brightness of a Lambertian surface to an observer is the same regardless of the observer's angle of view
        % El modelo de reflexi\'on difusa de Lambert representa un color ideal. El concepto de difusi\'on perfecta fue introducido
        % en el libro Photometria de Johann Heinrich Lambert\\

        % Aunque el modelo de Lambert \autocite{lambert} funciona perfectamente sobre materiales met\'alicos o pl\'asticos, el
        % modelo de Lambert produce sombras para valores altos del par\'ametro de rugosidad, lo que se acent\'ua al incrementar
        % el \'angulo de incidencia de la luz en la mayor\'ia de casos, al utilizar el modelo de Lmabert , sobre modelos de microfacetas
        % En 1994, el modelo de Oren-Nayar \autocite{orennayar} supuso el primer modelo de reflex\'on difusa basado en la teor\'ia de
        % microfacetas, especialmente en el trabajo de Torrance y Sparrow, optimizado m\'as tarde por Yoshiharu Gotanda \autocite{gotanda}
        % y posteriormente por
        % Disney presenta su propio modelo para la componente difusa \autocite{disney12}, mejorado en 2015 \autocite{disney15}
        % a new GGX+Smith diffuse shading model, a new and cheap approximation for Smith specular shadowing/masking \autocite{earlhammon}
        
        % Used Heitz Multiscattering simulator with GGX NDF. 
        % Using a multiscattering model means energy reflected diffusely can be reflected again by 
        % neighboring microfacets. 
        % We made some additions and some assumptions to his model:
        % 1) We treat each microfacet as a diffuse reflector of 100%.
        % 2) We apply Fresnel to determine energy reflected diffusely. 
        % 3) We use implicit F0 = 0.04 to determine energy reflected specularly and unable to 
        % participate in diffuse reflectance.
        % Interestingly, the properties we want fall out of this model automatically: 
        % Flattening for rough surfaces from strong grazing retroreflective response, 
        % Rounding for smooth surfaces from energy loss to specular\autocite{callofduty}
        % Oren and Nayar \autocite{orennayar} have proposed a diuse BRDF that models reectance from rough diuse surfaces. The model is based on microfacet theory,
        % specically on the work by Torrance and Sparrow. Oren and Nayar have proposed a diuse BRDF that models reflectance from rough diuse surfaces.
        % The model is based on microfacet theory, specically on the work by Torrance and Sparrow. El modelo de Gotanda \autocite{gotanda}
        
        % PBR:\\
        % Oren Nayar \autocite{orennayar}\\
        % Simplified Oren-Nayar \autocite{gotanda}\\
        % Disney Diffuse \autocite{disney12}\\
        % Renormalized Disney Diffuse \autocite{disney15}\\
        % Gotanda Diffuse \autocite{gotanda14}\\
        % PBR diffuse for GGX+Smith \autocite{earlhammon}\\
        % Multiscattering Diffuse \autocite{callofduty}\\

    \subsection{Componente especular}
    Aunuqe podemos considerar la mayor\'ia de materiales como isotr\'opicos, la radiancia de los puntos de su superficie no dependen
    del angulo de rotacion sobre la normal de la superficie, existen materiales sobre los que \'esta rotaci\'on s\'i afecta
    a la radiancia de salida. La mayor\'ia de modelos anal\'iticos presentan un modelo isotr\'opico, que es suficiente para
    representar una amplia variedad de materiales, sin embargo exiten alguno modelos que se encargan de representar \'estas
    variaciones en la normal de la superficie seg\'un la rotati\'on. A continuaci\'on se describen los diferentes modelos
    para cada t\'ermino del BRDF.

    % \todo[inline]{
    %     If we were to rotate the surface about its normal, the BRDF value (and consequently the
    %     resulting illumination) would remain unchanged. Materials with this characteristic such
    %     as smooth plastics have isotropic BRDFs.
    %     In general, most real-world BRDFs are anisotropic to some degree, but
    %     the notion of isotropic BRDFs is useful because many classes of analytical BRDF models
    %     fall within this class. In general, most real-world BRDFs are probably more isotropic
    %     than anisotropic though many real-world surfaces have subtle anisotropy. Any material
    %     that exhibits even the slightest anisotropic reflection has a BRDF that is anisotropic. 
    % }

        \subsubsection*{T\'ermino D}
            \todo[inline]{
                [Neubelt13] David Neubelt and Matt Pettineo. 2013. Crafting a Next-Gen Material Pipeline for The Order: 1886. Physically Based Shading in Theory and Practice, ACM SIGGRAPH 2013 Courses.
            }

            En el modelo de Cook-Torrance \autocite{cooktorrance} se utiliza el t\'ermino presentado en 1963 por Beckmann \autocite{beckmann}.
            Los modelos de Phong \autocite{phong} y Blinn-Phong \autocite{blinnphong}, a pesar de no ser basados en f\'isica, se pueden considerar
            como modelos emp\'iricos isotr\'opicos de la funci\'on de distribuci\'on de las normales.
            Recientemente, podemos destacar como modelos isotr\'opicos como los presentados por Disney \autocite{disney12}, el modelo
            de Eric Heitz \autocite{ggx} o el modelo de \textit{sheen} \autocite{sheenbrdf}, muy adecuado para la representaci\'on de tejidos
            y que explicaremos en detalle en el Cap\'itulo 5.\\
            \hspace*{1.5em}Por otra parte, el modelo anisotr\'opico de Phong \autocite{anisotropicphong} fue el primer modelo en representar
            \'este tipo de superficies, posteriormente, el modelo anistro\'opico de Beckmann \autocite{beckmannspinozo} o el de
            GGX \autocite{anisotropicggx} han trabajado tambi\'en sobre \'este tipo de materiales

            % \singlespacing
            % \todo[inline]{
            %     Isotropic\\
            %     Rotationally invariant (3D)\\
            %     True for many materials\\
            %     One dimension less\\
            % }

            % \todo[inline]{
            %     Anisotropic\\
            %     Depends on the angle of rotation around the surface normal\\
            % }
            
            % \singlespacing
            % Isotropic:\\
            % GGX \autocite{ggx}\\
            % Generalized Trowbridge Reitz (GTR) \autocite{disney12}\\
            % Ashikhmin \autocite{ashikhmin} <= check Filament\\
            % Estevez SheenBRDF \autocite{sheenbrdf} <= check Filament\\

            \todo[inline]{
                [McAuley15] Stephen McAuley. 2015. Rendering the World of Far Cry 4. GDC 2015, basado en [Revie12] Donald Revie. 2012. Implementing Fur in Deferred Shading. GPU Pro 2, Chapter 2.
            }

        \subsubsection*{T\'ermino F}
            El t\'ermino F es una aproximaci\'on de la funci\'on de Fresnel. La primera aproximaci\'on de de la funci\'on de Fresnel
            fue la presentada por Cook-Torrance. M\'as tarde Schlick \autocite{schlick}, Largarde \autocite{frostbite} y Gotanda \autocite{gotanda}
            presentaron sucesivas aproximaciones.

        \subsubsection*{T\'ermino G}
            El t\'ermino de geometr\'ia fue presentado en el modelo de microfacetas de Cook-Torrance \autocite{cooktorrance}
            y se utiliza para describir la cantidad de microfacetas que quedan en sombra y depende t\'ipicamente de los par\'ametros
            de rugosidad y distribuci\'on de las microfacetas. Adem\'as del citado modelo, tambi\'en son populares los modelos de
            Neumann \autocite{neumann}, y Kelemen \autocite{kelemen}. No obstante, en la actualidad, los modelos m\'as usados son los
            basados en el m\'etodo de Smith \autocite{smith}, que consiste en separa la f\'ormula en dos partes que utilizan la misma
            ecuaci\'on pero una utilizando el vector de vista y otra la de la luz, representando los fen\'omenos de sombreado y
            enmascaramiento respectivamente. De entre \'estos t\'erminos de geometr\'ia, cabe destacar el trabajo de Earl Hammon
            \autocite{earlhammon}, el modelo de Schlick-Beckmann \autocite{schlick}, o el utilizado en Unreal \autocite{unreal}, que remapea
            el valor de rugosidad para conseguir unos valores m\'as intuitivos para los artistas.
            
            % \singlespacing
            % Smith [1967]\\
            % Cook-Torrance [1982]
            % Neumann [1999]
            % Kelemen [2001]
            % Implicit [2013]

            % \singlespacing
            % Smith joint masking-shadowing function:\\
            % Smith-Beckmann\\
            % Smith-Schlick\\
            % Schlick-GGX\\
            % \todo[inline]{
            %     [Kelemen01] Csaba Kelemen et al. 2001. A Microfacet Based Coupled Specular-Matte BRDF Model with Importance Sampling. Eurographics Short Presentations.
            %     relacionado demuestra q no es PBR pero es una buena aproximacion para tiempo real [Heitz14] Eric Heitz. 2014. Understanding the Masking-Shadowing Function in Microfacet-Based BRDFs. Journal of Computer Graphics Techniques, 3 (2).
            % }

            % \todo[inline]{
            %     Blinn \autocite{blinn77}, Beckmann \autocite{beckmann}, Walter \autocite{ggx}, Neumann \autocite{neumann}, Kelemen
            %     \autocite{kelemen}, Smith \autocite{smith}, Karis \autocite{karis}, \autocite{reed}
            % }

        \subsubsection*{Luz indirecta en tiempo real}

            \textbf{Prefiltered Importance Sampling}
            [Krivanek08] Jaroslave Krivànek and Mark Colbert. 2008. Real-time Shading with Filtered Importance Sampling. Eurographics Symposium on Rendering 2008, Volume 27, Number 4.

            \textbf{Spherical Harmonics}
            [Ramamoorthi01] Ravi Ramamoorthi and Pat Hanrahan. 2001. On the relationship between radiance and irradiance: determining the illumination from images of a convex Lambertian object. Journal of the Optical Society of America, Volume 18, Number 10, October 2001.

            \textbf{Analytical DFG terms for IBL}
            \todo[inline]{
                [Karis14] Brian Karis. 2014. Physically Based Shading on Mobile. https://www.unrealengine.com/blog/physically-based-shading-on-mobile\\
                basado en: [Lazarov13] Dimitar Lazarov. 2013. Physically-Based Shading in Call of Duty: Black Ops. Physically Based Shading in Theory and Practice, ACM SIGGRAPH 2013 Courses.
            }
            \todo[inline]{
                [Narkowicz14] Krzysztof Narkowicz. 2014. Analytical DFG Term for IBL. https://knarkowicz.wordpress.com/2014/12/27/analytical-dfg-term-for-ibl
        }

    \subsection{Modelos de BRDF}
    Cook-Torrance \autocite{cooktorrance}
    Disney 2012 \autocite{disney12}
    Disney 2015 \autocite{disney15}
    Real Shading in Unreal 4 \autocite{karis}
    Electronic Arts \autocite{frostbite}
    Filament \autocite{filament}
    Especificaci\'on MaterialX \autocite{materialx}
    Referencia de t\'erminos del BRDF \autocite{brdfreference}

\chapter{Estado del arte}
En \'este cap\'itulo, por una parte se hace un breve repaso de los diferentes sistemas de almacenamiento m\'as comunes en aplicaciones web
y las diferencias entre ellos, que sirven para contextualizar la implementaci\'on del modelo de datos sobre una base de datos no
relacional basada en documentos que se describe en el Cap\'itulo 5.
Adem\'as, se comentan brevemente los avances en t\'ecnicas de visualizaci\'on realistas en motores de renderizado en tiempo real, que suponen
el marco te\'orico de este proyecto. Muchas de estas t\'ecnicas se analizan en detalle en el Cap\'itulo 4.

% \section{Arquitectura y almacenamiento en aplicaciones web}
% El patr\'on m\'as utilizado en la actualidad para la organizaci\'on de un proyecto web es el MVC, o modelo vista controlador.
% El modelo representa a la base de datos, mientras que el controlador es la aplicaci\'on en el lado del servidor que act\'ua
% como intermediaria entre las peticiones del usuario y las consultas a la base de datos y finalmente la vista es la visualizaci\'on
% que se le ofrece al usuario para permitirle interactuar con la aplicaci\'on.

\section{Sistemas de almacenamiento}
En la actualidad, los modelos de bases de datos m\'as utilizados son SQL \autocite{sql}, un sistema de filas y columnas
contenidas en tablas y relacionadas a trav\'es de claves y NoSQL, que agrupa a un conjunto de sistemas de datos no relacionales.
Mientras que las bases de datos relacionales aseguran la consistencia de los datos cumpliendo con las propiedades ACID
\autocite{acid}, los sistemas NoSQL nacieron durante los a\~nos 90 para dar respuesta a la alta disponibilidad de los recursos debido al creciente
n\'umero de usuarios de las aplicaciones web. Otras car\'acter\'isticas de los sistemas NoSQL son su flexibilidad en el modelo de datos
y su escalabilidad horizontal, al soportar estructuras distribu\'idas. Dentro de los sistemas no relacionales, podemos identificar
cuatro grupos principales: las bases de datos de clave/valor, en la sus datos se organizan en una estructura parecida a un diccionario
y ofrece una respuesta con muy baja latencia, bases de datos basadas en documentos, que relacionan una clave con una estructura
de datos compleja, bases de datos basadas en columnas, muy comunes en aplicaciones Big Data al funcionar muy bien sobre sistemas
distribu\'idos muy grandes y bases de datos basadas en grafos, que utilizan colecciones de nodos unidos por aristas que definen
las relaciones entre nodos.

        % \textbf{Graph databases}\\
        % Modeled after the concept of a mathematical graph, the graph database contains a collection of items (called nodes) and edges 
        % known as graphs) which define the relationships between items. Graph databases are designed for transactional integrity,
        % flexibility, and operational availability, making them a good choice for a range of use cases requiring speedy parsing of
        % relationships between heterogeneous data points – applications such as fraud prevention, advanced enterprise operations, online
        % transaction processing (OLTP), certain types of data security, churn analysis, and regulatory reporting, to name a few.

        % Their data models are a lot simpler than those of relational or other NoSQL databases. A drawback to widespread adoption, though,
        % is the lack of a standard graph database query language. You can’t use SQL (that’s what “NoSQL” means) or any other popular language.
        % Aspiring graph database developers should plan on acquiring fluency in a language only applicable to a few databases.\textbf{Neo4j}

        % \textbf{Key-value databases}\\
        % Another specialized non-relational type is the key-value database. Similar to a dictionary, it is used for fairly simply-organized
        % information that needs to be accessed with very low latency. A typical use case is the storage of configuration data. While these
        % lightweight databases are not that widely known, one was voted the “most loved database” in a popular annual survey for three years
        % running. \textbf{Redis}

        % \textbf{Document databases}\\
        % Document databases pair each key with a complex data structure known as a document. Documents can contain many different
        % key-value pairs, or key-array pairs, or even nested documents. MongoDB is a document database. \textbf{MongoDB}

        % \textbf{Column store databases}\\
        % The other NoSQL database type is known as a column store, wide column store, or column family database. Very fast and monstrously
        % scalable, these are capable of accommodating petabytes of data in vast distributed systems with thousands of servers. All of this
        % makes them perfect for specific kinds of Big Data applications.\textbf{Cassandra}

    % El modelo de base de datos no relacional fue introducido  por Edgar F. Codd \autocite{sql}, utilizando un sistema de filas y
    % columnas asociadas utilizando las claves de las filas. SQL ha sido usado y es usado extensivamente desde su nacimiento,
    % sin embargo, las nuevas necesidades de las aplicaciones durante los a\~nos 90 el nacimiento de nuevos sistemas de informaci\'on
    % las bases de datos no relacionales, que permiten una alta disponibilidad de los datos, a coste de no cumplir las propiedades ACID
    % frente a la consistencia de las bases de datos relacionales, las NoSQL aportan flexibilidad en sus estructuras de datos,
    % escalabidad horizontal, al soportar estructuras distribuidas y una alta disponibilidad de los datos

\section{T\'ecnicas de visualizaci\'on realista en tiempo real}

En 1970 se presenta el primer algoritmo de sombreado en gr\'aficos rasterizados \autocite{gouraud}, posteriormente 
Phong \autocite{phong} presentada un modelo para la componente especular, revisado y mejorado posteriormente por James Blinn
\autocite{blinnphong}. Estos modelos consiguen describir el comportamiento del material en todos sus puntos,
estableciendo un ratio entre la luz que llega a una superficie y la luz reflejada, sin embargo no tienen en cuenta el modelo f\'isico de
la luz, por lo que no podemos hablar de materiales PBR hasta 1981, a\~no en el que Robert Cook y Kenneth Torrance presentan su modelo
de BRDF basado en microfacetas \autocite{cooktorrance}

    \subsection{Componentes y t\'erminos del BRDF basado en microfacetas}
    % \todo[inline]{
    %     Blinn \autocite{blinn77}, Beckmann \autocite{beckmann}, Walter \autocite{ggx}, Neumann \autocite{neumann}, Kelemen
    %     \autocite{kelemen}, Smith \autocite{smith}, Karis \autocite{karis}, \autocite{reed}
    % }

    El modelo de Cook-Torrance \autocite{cooktorrance}, que explicaremos en detalle en el Cap\'itulo 4, sus principales aportaciones son
    separar el BRDF como la suma de sus dos componentes, especular y difuso y utilizar un BRDF especular basado en microfacetas. El BRDF
    especular es la como la combinaci\'on de los t\'erminos de distribuci\'on de las normales, el de geometr\'ia o visibilidad que modela
    los fen\'omenos de sombra y enmascaramiento y el de Fresnel.

        \subsubsection{Componente difusa}
            Johann Heinrich Lambert introdujo en 1760 \autocite{lambert} el concepto de difusi\'on perfecta, superficies cuya radiancia no
            dependen del punto de vista, si no que parecen tener la misma radiancia desde todas las direcciones. Este modelo fue el utilizado
            por Cook-Torrance \autocite{cooktorrance} y se sigue utilizando con frecuencia a d\'ia de hoy, ya que supone una buena
            aproximaci\'on para una gran parte de materiales. Sin embargo, este modelo ideal no tiene en cuenta la conservaci\'on
            de la energ\'ia y cuando se utiliza en conjunto con un BRDF basado en f\'isica, podemos observar un oscurecimiento
            en las zonas de mayor rugosidad, que se acent\'ua a medida que aumenta al incrementar el \'angulo de incidencia de la luz.
            El modelo de Oren-Nayar \autocite{orennayar}, presenta por primera vez un t\'ermino difuso f\'isicamente plausible,
            basado en la teor\'ia de microfacetas, especialmente en el trabajo de Torrance-Sparrow \autocite{torrancesparrow}. M\'as
            tarde, los modelos de Disney \autocite{disney12} \autocite{disney15}, Yoshiharu Gotanda \autocite{gotanda14} o Eal Hammon
            \autocite{earlhammon} presentaron sucesivas aproximaciones a la componente difusa basada en f\'isica teniendo en cuenta la
            distribuci\'on de las microfacetas dependiendo de los t\'erminos utilizados para la componente especular.

        \subsubsection{Componente especular}
        Aunuqe podemos considerar la mayor\'ia de materiales como isotr\'opicos, la radiancia de los puntos de su superficie no dependen
        del angulo de rotacion sobre la normal de la superficie, existen materiales sobre los que \'esta rotaci\'on s\'i afecta
        a la radiancia de salida. La mayor\'ia de modelos anal\'iticos presentan un modelo isotr\'opico, que es suficiente para
        representar una amplia variedad de materiales, sin embargo exiten alguno modelos que se encargan de representar \'estas
        variaciones en la normal de la superficie seg\'un la rotati\'on. A continuaci\'on se describen los diferentes modelos
        para cada t\'ermino del BRDF.

            \paragraph*{\hspace*{1.5em}T\'ermino D}
                En el modelo de Cook-Torrance \autocite{cooktorrance} se utiliza el t\'ermino presentado en 1963 por Beckmann \autocite{beckmann}.
                Los modelos de Phong \autocite{phong} y Blinn-Phong \autocite{blinnphong}, a pesar de no ser basados en f\'isica, se pueden considerar
                como modelos emp\'iricos isotr\'opicos de la funci\'on de distribuci\'on de las normales.
                Recientemente, podemos destacar como modelos isotr\'opicos como los presentados por Disney \autocite{disney12}, el modelo
                de Eric Heitz \autocite{ggx} o el modelo de \textit{sheen} \autocite{sheenbrdf}, muy adecuado para la representaci\'on de tejidos
                y que explicaremos en detalle en el Cap\'itulo 5.\\
                \hspace*{1.5em}Por otra parte, el modelo anisotr\'opico de Phong \autocite{anisotropicphong} fue el primer modelo en representar
                \'este tipo de superficies, posteriormente, el modelo anistro\'opico de Beckmann \autocite{beckmannspinozo} o el de
                GGX \autocite{anisotropicggx} han trabajado tambi\'en sobre \'este tipo de materiales

                \todo[inline]{
                    Estevez and Kulla propose a different NDF (called the “Charlie” sheen) that is based on an exponentiated
                    sinusoidal instead of an inverted Gaussian. This NDF is appealing for several reasons: its parameterization
                    feels more natural and intuitive, it provides a softer appearance and, as shown in equation 49, its
                    implementation is simpler:
                }

            \paragraph*{\hspace*{1.5em}T\'ermino F}
                El t\'ermino F es una aproximaci\'on de la funci\'on de Fresnel. La primera aproximaci\'on de de la funci\'on de Fresnel
                fue la presentada por Cook-Torrance. M\'as tarde Schlick \autocite{schlick}, Largarde \autocite{frostbite} y Gotanda \autocite{gotanda}
                presentaron sucesivas aproximaciones.

            \paragraph*{\hspace*{1.5em}T\'ermino G}
                El t\'ermino de geometr\'ia fue presentado en el modelo de microfacetas de Cook-Torrance \autocite{cooktorrance}
                y se utiliza para describir la cantidad de microfacetas que quedan en sombra y depende t\'ipicamente de los par\'ametros
                de rugosidad y distribuci\'on de las microfacetas. Adem\'as del citado modelo, tambi\'en son populares los modelos de
                Neumann \autocite{neumann}, Kelemen \autocite{kelemen} o el presentado de por Neubelt \autocite{theordertalk}. No
                obstante, en la actualidad, los modelos m\'as usados son los basados en el m\'etodo de Smith \autocite{smith}, que
                consiste en separar la f\'ormula en dos partes que utilizan la misma ecuaci\'on pero una utilizando el vector de vista
                y otra la de la luz, representando los fen\'omenos de sombreado y enmascaramiento respectivamente. De entre \'estos t\'erminos
                de geometr\'ia, cabe destacar el trabajo de Earl Hammon \autocite{earlhammon}, el modelo de Schlick-Beckmann \autocite{schlick},
                o el utilizado en Unreal \autocite{unreal}, que remapea el valor de rugosidad para conseguir unos valores m\'as intuitivos para
                los artistas.
                
        \subsection{Luz indirecta en tiempo real}
        La ecuaci\'on para el c\'alculo de iluminaci\'on indirecta, analizada en detalle en el Cap\'itulo 4, fue presentada por primera vez
        por Kajiya \autocite{kajiya}, sin embargo la parte integral de la ecuaci\'on supone un c\'aculo recursivo que no puede ser resulto
        en tiempo real. Es por ello, que para los motores de renderizado en tiempo real, se utilizan t\'ecnicas que permiten conseguir
        \'este efecto, a trav\'es de precalculos o aproximaciones anal\'iticas.

            \subsubsection*{\hspace*{1.5em}Prec\'aculo de irradiancia sobre un punto}
            Para calcular la irradiancia sobre un punto debida al entorno, Paul Debevec \autocite{debevec} present\'o la t\'ecnica
            de mapas de irradiancia, mientras que en 2001, Ravi Ramamoorthi y Pat Hanrahan \autocite{sh} presentan la t\'ecnica de esf\'ericos har\'omicos,
            que permite la compresi\'on de informaci\'on utilizando una representaci\'on de frecuencias de color frente a espacio.
            La t\'ecnica de \textit{pre-filterd importance sampling} \autocite{prefilteredimportancesampling}, muy parecida a la de mapas
            de irradiancia \autocite{debevec}, pero adem\'as utiliza la t\'ecnica de \textit{mipmapping} para devolver diferentes mapas
            de entorno en funci\'on del valor de rugosidad del material.

            \subsubsection*{\hspace*{1.5em}Integraci\'on de los t\'erminos DFG}
            La aproximaci\'on de sumas parciales, \autocite{karis} almacena el resultado de la integraci\'on de los t\'erminos en un \textit{look-up texture}
            y permite simplemente consultarlos en tiempo real. Adem\'as existen funciones anal\'iticas seg\'un el tipo de BRDF,
            como la utizada por Epic Games \autocite{dfgapproximation} para plataformas m\'oviles o la presentada por Krzysztof Narkowicz
            \autocite{narkowicz}, que resultan una buena aproximaci\'on en dispositivos con un rendimiento limitado.

        \subsection{Modelos de referencia de BRDF en la actualidad}
        Hoy d\'ia el modelo de referencia es el de Disney \autocite{disney12} \autocite{disney15}, conocido como \textit{"Principled"},
        y que explicaremos en detalle en el Cap\'itulo 4. Adem\'as los modelos de Electronic Arts \autocite{frostbite} o Filament \autocite{filament}
        suponen implementaciones completas de modelos de BRDDF que integran iluminaci\'on global y  que sirven de inspiraci\'on a otros
        motores de renderizado en tiempo real.

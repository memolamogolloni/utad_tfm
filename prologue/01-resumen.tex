\chapter{Res\'umen}

El realismo de los motores de renderizado web ha aumentado constantemente a lo largo de los \'ultimos a\~nos gracias al incremento en
capacidad de computaci\'on de los ordenadores, el soporte de APIs gr\'aficas sobre el navegador y el desarrollo de librer\'ias
que integran t\'ecnicas de renderizado en tiempo real que acercan a los desarrolladores web, de una forma accesible,
herramientas que antes estaban destinadas al mundo de los videojuegos o la simulaci\'on. Motivado por \'este desarrollo t\'ecnico,
las plataformas web buscan sorprender a sus usuarios con nuevas formas de contenido, lo que ha propiciado recientemente
el aumento de \'estes librer\'ias de renderizado 3D: ThreeJs, Babylon, AFrame o Filament son algunos ejemplos de proyectos en cont\'inuo
crecimiento. El prop\'osito de \'estas librer\'ias es ofrecer una experiencia de uso c\'omoda para los desarrolladores, por lo que intentan
abstraer de los detalles de implementaci\'on de bajo nivel de la API gr\'afica, proporcionando soluciones que funcionan muy bien
para la mayor parte de casos a costa de un menor control sobre el sistema de renderizado.\\

En el presente trabajo se extiende la librer\'ia de materiales de una de las citadas librer\'ias, ThreeJs, con intenci\'on
de crear un nuevo material que permita representar los fen\'omenos f\'isicos caracter\'isticos de los tejidos y que no se
modelan los materiales est\'andar de la librer\'ia para conseguir de \'esta forma un mayor grado de realismo sobre este tipo
de material.

\chapter{Resumen}

Con la llegada del API gr\'afica WebGL y posterioremente WebGL 2.0 han aparecido librer\'ias de alto nivel, como ThreeJs, Babylon,
AFrame, Filament, etc.,  que proporcionan a los desarrolladores un interfaz que cubre la mayor parte de casos de uso
del d\'ia a d\'ia y ofrece una experiencia de desarrollo intuitiva a costa de ocultar detalles de implementaci\'on de bajo nivel.\\
% Las necesidades espec\'ificas de este proyecto, la visualizaci\'on realista de tejidos sobre un entorno web, hace necesario un
% estudio de las diferentes tipos de t\'ecnicas de r realista para representar este tipo de material, especialmente complejo,
% ya que las soluciones 

No obstante, el cometido de este proyecto es renderizar tejidos sobre un entorno web interactivo de la forma m\'as realista posible
y los modelos est\'andar basados en f\'isica no consiguen, por lo general, una representaci\'on lo suficientemente
precisa para esta labor. Las im\'agenes generadas por este motor de renderizado sirven como previsualizaci\'on de im\'agenes
en alta fidelidad generadas por un motor de trazado de rayos, por lo que, adem\'as de realismo, debe buscarse la mayor
coherencia posible entre estos dos motores.\\

% interactivo se compara\'n con im\'agenes de alta
% fidelidad generadas por un motor basado en trazado de rayos, por lo que, adem\'as de realismo, debe buscarse la mayor
% coherencia posible entre estos dos motores.\\

Basado en el estudio te\'orico de los fundamentos t\'ecnicos del renderizado basado en aspectos f\'isicos (PBR) y el an\'alisis
de diferentes modelos de Bidirectional Reflectance Distribution Function (BRDF) para la representaci\'on de materiales,
este trabajo presenta un material especializado en tejidos integrado dentro de una arquitectura de servicios en la nube.
Para dar soporte a este nuevo material en los servicios en la nube, se han extendido el modelo de datos de los materiales
y desarrollado la interfaz que propociona acceso a ellos. El modelo de BRDF para telas se ha implementado sobre ThreeJs,
extendiendo la librer\'ia con un nuevo material que ofrece la misma interfaz que los materiales nativos de la librer\'ia.\\

Los resultados presentados ofrecen comparativas entre el material de telas y el material PBR est\'andar de ThreeJs
y entre las im\'agenes generadas por este nuevo modelo e im\'agenes de referencia renderizadas por el motor de
trazado de rayos. En la comparativa con el material de ThreeJs se aprecia un mayor grado de realismo en el nuevo modelo, adem\'as
frente a las im\'agenes con el motor de trazados de rayos se aprecia poca discrepancia.


% muestran una comparativa del nuevo material frente a los materiales est\'andar de la librer\'ia en los que
% se muestra una mayor fidelidad de los im\'agenes obtenidas con este nuevo modelo.

% Adem\'as las im\'agenes presentadas
% frente a frente con los resultados generados por el motor de renderizado basado en trazado de rayos, que muestran una mayor
% coherencia entre ambos motores.


% Sin embargo, los tejidos son materiales dif\'iciles de simular al tratarse de estructuras compuestas por gran cantidad de elementos
% cuyas propiedades e interacciones entre s\'i afectan al aspecto de la superficie 
% y que se compar\'an con los resultados obtenidos por un motor de trazado de rayos
% El motor de renderizado utilizado en el cliente web de Author es ThreeJs,



% Esta integraci\'on es fruto del estudio de diferentes modelos de BRDFs, con el fin de enteder las bases teor\'ias sobre la que se sustentan los
% motores de renderizado PBR, y el estudio de los detalles de implementaci\'on a bajo nivel del sistema de materiales de ThreeJs,
% el motor de renderizado utilizado en Author.\\
% Para ofrecer soporte a este nuevo tipo de material, se han actualizado el modelo
% de datos y las interfaces de los servicios en la nube de Seddi, adem\'as de extender la librer\'ia ThreeJs de forma que el nuevo
% material presente la misma interfaz que los materiales nativos de la librer\'ia.\\



% En este trabajo se actualizan el modelo de datos e interfaces de los servicios en la nube de Seddi, para proporcionar la
% infraestructura necesaria que soporte un nuevo tipo de material especializado en tejidos dentro del contexto de Author. Adem\'as
% del desarrollo de infraestructura, se estudian diferentes modelos de BRDF para entender la teor\'ia sobre la que se sostienen los motores de renderizado basados en f\'isica y
% se analiza la arquitectura del motor de renderizado de ThreeJs

% y se es
% supone un an\'alisis de los detalles de implementaci\'on a bajo nivel del sistema de materiales en ThreeJs, el motor de renderizado
% usado en Author, para extenderlo implementando un nuevo Bidirectional Reflectance Distribution Function (BRDF) que represente
% adecuadamente los fen\'omenos f\'isicos caracter\'isticos de este tipo de material.\\


% web para 

% Este trabajo ampl\'ia el modelo de datos y actualiza las interfaces de los servicios web para dar soporte a un tipo de material
% especial para tejidos q

% Este trabajo supone la comprension a bajo nivel de los detalles de implementacion de la libreria de materiales de ThreeJs
% para extenderla y crear un material especifico para telas.

% El realismo de los motores de renderizado ha aumentado constantemente a lo largo de los \'ultimos a\~nos gracias al incremento en
% capacidad de computaci\'on de los ordenadores y nuevas t\'ecnicas de renderizado en tiempo real basadas en f\'isica.
% Gracias al soporte en web de la API gr\'afica WebGL y m\'as recientemente WebGL 2.0, estas t\'ecnicas, se pueden ver a d\'ia de hoy
% en aplicaciones web, que dotan a sus gr\'aficos de una calidad excepcional, antes exclusiva del sector videojuegos.
% Librer\'ias como ThreeJs, Babylon o Filament, acercan muchas de estas t\'ecnicas a disposici\'on del desarrollador,
% ofreciendo una API intuitiva, que oculta detalles de implementaci\'on de bajo nivel para conseguir una experiencia de desarrollo
% intuitiva. Estas librer\'ias cubren gran parte de los casos de uso del d\'ia a d\'ia, no obstante, las necesidades espec\'ificas
% en el departamento de Author hacen necesario extender la librer\'ia para dotarla de nuevas funcionalidades.\\
% Este trabajo extiende la librer\'ia de 3D ThreeJs, una de las m\'as populares del entorno web, para
% crear un nuevo material que permita representar los fen\'omenos f\'isicos caracter\'isticos de cierto tipo de tejidos y que no se
% modelan los materiales est\'andar de la librer\'ia, tratando de conseguir un mayor grado de realismo sobre este tipo
% de materiales.


% mayor\'ia de casos de uso, pero requieren implementaciones espec\'ificas para
% dar soporte a casos de uso concretos como el de este proyecto: a\~nadir 

% ocultando
% detalles de implementaci\'on de bajo nivel para , que cumple con la
% mayor\'ia de casos de uso, 

% bajo
% unas APIs intuitivas que ocultan detalles de implementaci\'on de bajo nivel.

% ha incrementado la cantidad
% y calidad del contenido 3D en el navegador, popularizando librer\'ias como ThreeJs, Babylon o AFrame, que acercan estas nuevas t\'enicas
% de renderizado, antes limitadas esencialmente al desarrollo de videojuegos, que buscan ofrecer
% una experiencia c\'omoda de desarrollo y 

% abstrayendo detalles de implementaci\'on y ofreciendo 

% que acercan a los desarrolladores web, de una forma accesible,
% herramientas que antes estaban destinadas al mundo de los videojuegos o la simulaci\'on. Motivado por \'este desarrollo t\'ecnico,
% las plataformas web buscan sorprender a sus usuarios con nuevas formas de contenido, lo que ha propiciado recientemente
% el aumento de \'estes librer\'ias de renderizado 3D: ThreeJs, Babylon, AFrame o Filament son algunos ejemplos de proyectos en cont\'inuo
% crecimiento. El prop\'osito de \'estas librer\'ias es ofrecer una experiencia de uso c\'omoda para los desarrolladores, por lo que intentan
% abstraer de los detalles de implementaci\'on de bajo nivel de la API gr\'afica, proporcionando soluciones que funcionan muy bien
% para la mayor parte de casos a costa de un menor control sobre el sistema de renderizado.\\

% En el presente trabajo se extiende la librer\'ia de materiales de una de las citadas librer\'ias, ThreeJs, con intenci\'on
% de crear un nuevo material que permita representar los fen\'omenos f\'isicos caracter\'isticos de los tejidos y que no se
% modelan los materiales est\'andar de la librer\'ia para conseguir de \'esta forma un mayor grado de realismo sobre este tipo
% de material.

% El realismo de los motores de renderizado ha aumentado constantemente a lo largo de los \'ultimos a\~nos gracias al incremento en
% capacidad de computaci\'on de los ordenadores, el soporte de APIs gr\'aficas sobre el navegador y el desarrollo de librer\'ias
% que integran t\'ecnicas de renderizado en tiempo real que acercan a los desarrolladores web, de una forma accesible,
% herramientas que antes estaban destinadas al mundo de los videojuegos o la simulaci\'on. Motivado por \'este desarrollo t\'ecnico,
% las plataformas web buscan sorprender a sus usuarios con nuevas formas de contenido, lo que ha propiciado recientemente
% el aumento de \'estes librer\'ias de renderizado 3D: ThreeJs, Babylon, AFrame o Filament son algunos ejemplos de proyectos en cont\'inuo
% crecimiento. El prop\'osito de \'estas librer\'ias es ofrecer una experiencia de uso c\'omoda para los desarrolladores, por lo que intentan
% abstraer de los detalles de implementaci\'on de bajo nivel de la API gr\'afica, proporcionando soluciones que funcionan muy bien
% para la mayor parte de casos a costa de un menor control sobre el sistema de renderizado.\\

% En el presente trabajo se extiende la librer\'ia de materiales de una de las citadas librer\'ias, ThreeJs, con intenci\'on
% de crear un nuevo material que permita representar los fen\'omenos f\'isicos caracter\'isticos de los tejidos y que no se
% modelan los materiales est\'andar de la librer\'ia para conseguir de \'esta forma un mayor grado de realismo sobre este tipo
% de material.

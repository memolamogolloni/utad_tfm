\chapter{Conceptos b\'asicos}

\section{Unidades b\'asicas radiom\'etricas}
\bgroup
    Cuando hablamos de la cantidad de luz, no es la medicion sobre un foton de luz independiente, si no que se tratan de medidas en relacion
    al tiempo, la direccion o el \'area.

    \begin{itemize}
    \item[] \textbf {{\'Angulo s\'olido}}
    \item[] \textbf {Flujo} La unidad que representa la energia sobre la unidad de tiempo es el flujo, representado por Phi. Es la energía que transportan las ondas
    por unidad de tiempo, se mide en vatios. En los motores de render se utiliza para expresar la cantidad total de energia emitida por un a fuente de luz.
    \begin{equation}
        \Phi_e = \dfrac{d{Q_e}}{dt}
    \end{equation}
    \item[] \textbf {Intensidad}
    \begin{equation}
        I = \dfrac{d\Phi}{d\omega}
    \end{equation} 
    \item[] \textbf {Irradiancia} La irradiancia, es la cantidad de energ\'ia por unidad de tiempo por unidad de superficie, o flujo por superficie. Se representa como
    E, su unidad son los vatios/m2 y se utiliza para medir la cantidad de luz que incide sobre una superficie.
    \begin{equation}
        E = \dfrac{d\Phi}{dA}
    \end{equation}
    Cuando este flujo de radiancia se mide en direcci\'on contraria, de salida, se llama emitancia (M)
    \item[] \textbf {Radiancia} Simula luz tan lejana que sus rayos son paralelos entre si, como por ejemplo el sol.
    \begin{equation}
        L = \dfrac{d^2\Phi}{dA_{proj}d\omega}
    \end{equation}
    \end{itemize}
\egroup


\section{BxDF}
    \bgroup
    A continuaci\'on se detallan los nombres de las funciones en funcion del fen\'omeno f\'isico que modelan.
    \begin{itemize}
        \item[] \textbf {BRDF} Es la funci\'on que modela el comportamiento de la luz al golpear una superficie opaca, la reflexi\'on. Fue definido por primera vez en 1965 por
        Fred Nicodemus y su definici\'on es el ratio entre radiancia reflectada e irraciancia incidente.
        Para determinar el \'angulo de salida del rayo se utiliza la ley de la reflexi\'on y las tres caracter\'isticas que ha de cumplir un BRDF basado en
        f\'isica es que sea positivo, que cumpla con la reciprocidad de Helmholtz y que cumpla con la ley de conservaci\'on de la energ\'ia.
        \item[] \textbf {BTDF} Describe el comportamiento del rayo de luz al atravesar una superficie, la refracci\'on. Para calcular el angulo de salida del rayo se utiliza la
        ley de Snell, y al contrario que el BRDF, no sumple el principio de reciprocidad de Helmholtz.
        \item[] \textbf {BSSRDF y BSSTF} Son ampliaciones del modelo de reflexi\'on y refracci\'on, respectivamente, teniendo en cuenta las reflexiones internas del rayo a traves de la
        superficie del objeto.
        \item[] \textbf {BSDF} Se utiliza comunmente para hablar de cualquier forma de BxDF. En un sentido mas estricto, se refiere al conjunto de un BSSRDF y un BSSTDF.
    \end{itemize}
    \egroup

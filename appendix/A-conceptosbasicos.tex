\chapter{Conceptos b\'asicos}

\section{Unidades b\'asicas radiom\'etricas}
\todo[inline]{Cambiar/completar}

\bgroup
    Cuando hablamos de la cantidad de luz, no es la medicion sobre un foton de luz independiente, si no que se tratan de medidas en relacion
    al tiempo, la direccion o el \'area.

    \begin{itemize}
    \item[] \textbf {{\'Angulo s\'olido}}
    \item[] \textbf {Flujo} La unidad que representa la energia sobre la unidad de tiempo es el flujo, representado por Phi. Es la energía que transportan las ondas
    por unidad de tiempo, se mide en vatios. En los motores de render se utiliza para expresar la cantidad total de energia emitida por un a fuente de luz.
    \begin{equation}
        \Phi_e = \dfrac{d{Q_e}}{dt}
    \end{equation}
    \item[] \textbf {Intensidad}
    \begin{equation}
        I = \dfrac{d\Phi}{d\omega}
    \end{equation} 
    \item[] \textbf {Irradiancia} La irradiancia, es la cantidad de energ\'ia por unidad de tiempo por unidad de superficie, o flujo por superficie. Se representa como
    E, su unidad son los vatios/m2 y se utiliza para medir la cantidad de luz que incide sobre una superficie.
    \begin{equation}
        E = \dfrac{d\Phi}{dA}
    \end{equation}
    Cuando este flujo de radiancia se mide en direcci\'on contraria, de salida, se llama emitancia (M)
    \item[] \textbf {Radiancia} Simula luz tan lejana que sus rayos son paralelos entre si, como por ejemplo el sol.
    \begin{equation}
        L = \dfrac{d^2\Phi}{dA_{proj}d\omega}
    \end{equation}
    \end{itemize}
\egroup

\section{BxDF}
    \bgroup
    A continuaci\'on se detallan los nombres de las funciones en funcion del fen\'omeno f\'isico que modelan.
    \begin{itemize}
        \item[] \textbf {BRDF} Es la funci\'on que modela el comportamiento de la luz al golpear una superficie opaca, la reflexi\'on. Fue definido por primera vez en 1965 por
        Fred Nicodemus y su definici\'on es el ratio entre radiancia reflectada e irraciancia incidente.
        Para determinar el \'angulo de salida del rayo se utiliza la ley de la reflexi\'on y las tres caracter\'isticas que ha de cumplir un BRDF basado en
        f\'isica es que sea positivo, que cumpla con la reciprocidad de Helmholtz y que cumpla con la ley de conservaci\'on de la energ\'ia.
        \item[] \textbf {BTDF} Describe el comportamiento del rayo de luz al atravesar una superficie, la refracci\'on. Para calcular el angulo de salida del rayo se utiliza la
        ley de Snell, y al contrario que el BRDF, no sumple el principio de reciprocidad de Helmholtz.
        \item[] \textbf {BSSRDF y BSSTF} Son ampliaciones del modelo de reflexi\'on y refracci\'on, respectivamente, teniendo en cuenta las reflexiones internas del rayo a traves de la
        superficie del objeto.
        \item[] \textbf {BSDF} Se utiliza comunmente para hablar de cualquier forma de BxDF. En un sentido mas estricto, se refiere al conjunto de un BSSRDF y un BSSTDF.
    \end{itemize}
    \egroup

    \section{Integraci\'on con ThreeJs}
    La soluci\'on elegida para extender el sistema de materiales de Three ha sido crear un fork, extendiendo la
    librer\'ia para implementar las funcionalidades necesarias que dan soporte a estos nuevos motores de shading.
    Siguiendo la nomenclatura, de ThreeJs (MeshStandardMaterial, MeshPhysicalMaterial, etc) se ha creado un material
    MeshClothMaterial, basado en los material Cloth de Filament.\\
    ThreeJs utiliza un sistema de chunks (trozos) se componen en tiempo de ejecuci\'on para acabar formando los vertex
    y fragment shaders que se utilizan en los programas de WebGL. Los chunks se componen en la libreria de shaders, la
    clase ShaderLib.
    Para crear nuestro MeshClothMaterial, hemos de extender de la clase base Material, de la que extienden el resto de
    materiales. En este caso, de la misma forma que hace MeshPhysicalMaterial, extenderemos de MeshStandardMaterial,
    que dispone de la mayor parte de uniforms y attributes que necesita nuestro shader.\\

    \bgroup

    \begin{lstlisting}[caption=Clase MeshClothMaterial]
import { Vector2 } from '../math/Vector2.js';
import { MeshStandardMaterial } from './MeshStandardMaterial.js';
import { Color } from '../math/Color.js';
import brdfCloth from './clothBRDF.js';

/**
    * parameters = {
    *  reflectivity: <float>,
    *
    *  sheen: <Color>,
    *
    *  transmission: <float>,
    *  transmissionMap: new THREE.Texture( <Image> ),
    *
    *  subsurface: <Vector3>,
    * }
    */

function MeshClothMaterial( parameters ) {

    MeshStandardMaterial.call( this );

    this.defines = {

    'STANDARD': '',
    'CLOTH': ''

    };

    this.type = 'MeshClothMaterial';
    this.sheen = null;

    this.transmission = 0.0;
    this.transmissionMap = null;
    this.subsurface = null;

    this.brdfCloth = brdfCloth;

    this.setValues( parameters );

}

MeshClothMaterial.prototype = Object.create( MeshStandardMaterial.prototype );
MeshClothMaterial.prototype.constructor = MeshClothMaterial;

MeshClothMaterial.prototype.isMeshClothMaterial = true;

MeshClothMaterial.prototype.copy = function ( source ) {

    MeshStandardMaterial.prototype.copy.call( this, source );

    this.defines = {

    'STANDARD': '',
    'CLOTH': ''

    };

    if ( source.sheen ) {

    this.sheen = ( this.sheen || new Color() ).copy( source.sheen );

    }

    this.transmission = source.transmission;
    this.transmissionMap = source.transmissionMap;

    if ( source.subsurface ) {

    this.subsurface = ( this.subsurface || new Color() ).copy( source.subsurface );

    } else {

    this.subsurface = null;

    }

    if ( source.brdfCloth ) {

    this.brdfCloth = source.brdfCloth;

    } else {

    this.brdfCloth = null;

    }

    return this;

};

export { MeshClothMaterial };
        
    \end{lstlisting}
    
    Para que el motor de render de ThreeJs reconozca este nuevo material es necesario, a\~nadir
    su tipo (MeshClothMaterial) al mapa de ShaderIds para que el gestor de programas (WebGLPrograms)
    lo utilice para detectar obtener los uniforms y shaders necesarios para el material. Adem\'as
    los nuevos par\'ametros necesarios para el material se deben de incluir en el array
    parametersNames, de forma que el sistema de cacheo de programas de ThreeJs detecte estas
    nuevas propiedades.\newline
    
    \begin{lstlisting}[caption=Clase MeshClothMaterial]
function WebGLPrograms( renderer, cubemaps, extensions, capabilities, bindingStates, clipping ) {

// ...

const shaderIDs = {
    MeshDepthMaterial: 'depth',
    MeshDistanceMaterial: 'distanceRGBA',
    MeshNormalMaterial: 'normal',
    MeshBasicMaterial: 'basic',
    MeshLambertMaterial: 'lambert',
    MeshPhongMaterial: 'phong',
    MeshToonMaterial: 'toon',
    MeshStandardMaterial: 'physical',
    MeshPhysicalMaterial: 'physical',
    MeshClothMaterial: 'cloth',
    MeshMatcapMaterial: 'matcap',
    LineBasicMaterial: 'basic',
    LineDashedMaterial: 'dashed',
    PointsMaterial: 'points',
    ShadowMaterial: 'shadow',
    SpriteMaterial: 'sprite'
};

const parameterNames = [
    "precision", "isWebGL2", "supportsVertexTextures", "outputEncoding", "instancing", "instancingColor",
    "map", "mapEncoding", "matcap", "matcapEncoding", "envMap", "envMapMode", "envMapEncoding", "envMapCubeUV",
    "lightMap", "lightMapEncoding", "aoMap", "emissiveMap", "emissiveMapEncoding", "bumpMap", "normalMap", "objectSpaceNormalMap", "tangentSpaceNormalMap", "clearcoatMap", "clearcoatRoughnessMap", "clearcoatNormalMap", "displacementMap", "specularMap", "subsurface", "brdfCloth",
    "roughnessMap", "metalnessMap", "gradientMap",
    "alphaMap", "combine", "vertexColors", "vertexTangents", "vertexUvs", "uvsVertexOnly", "fog", "useFog", "fogExp2",
    "flatShading", "sizeAttenuation", "logarithmicDepthBuffer", "skinning",
    "maxBones", "useVertexTexture", "morphTargets", "morphNormals",
    "maxMorphTargets", "maxMorphNormals", "premultipliedAlpha",
    "numDirLights", "numPointLights", "numSpotLights", "numHemiLights", "numRectAreaLights",
    "numDirLightShadows", "numPointLightShadows", "numSpotLightShadows",
    "shadowMapEnabled", "shadowMapType", "toneMapping", 'physicallyCorrectLights',
    "alphaTest", "doubleSided", "flipSided", "numClippingPlanes", "numClipIntersection", "depthPacking", "dithering",
    "sheen", "transmissionMap"
];

// ...

function getParameters( material, lights, shadows, scene, object ) {

    const shaderID = shaderIDs[ material.type ];

    // ...

    let vertexShader, fragmentShader;

    if ( shaderID ) {

    const shader = ShaderLib[ shaderID ];

    vertexShader = shader.vertexShader;
    fragmentShader = shader.fragmentShader;

    }

    // ...

    const parameters = {

    shaderID: shaderID,

    vertexShader: vertexShader,
    fragmentShader: fragmentShader,

    sheen: !! material.sheen,

    subsurface: !! material.subsurface,

    brdfCloth: !! envMap && shaderID === 'MeshClothMaterial',
    // ...
    };

    // ...

    return parameters;
}
    \end{lstlisting}

    Finalmente, en el gestor de materiales de ThreeJs, necesitamos a\~nadir un nuevo m\'etodo
    que actualice los unfiroms del programa creado, de la misma forma que se hace con los
    materiales nativos de la librer\'ia.\newline
    
    \begin{lstlisting}[caption=Clase WebGLMaterials]
function refreshMaterialUniforms( uniforms, material, pixelRatio, height ) {

    // ...

    if ( material.isMeshStandardMaterial ) {

    refreshUniformsCommon( uniforms, material );

    if ( material.isMeshPhysicalMaterial ) {

        refreshUniformsPhysical( uniforms, material );

    } else if ( material.isMeshClothMaterial ) {

        refreshUniformsCloth( uniforms, material );

    } else {

        refreshUniformsStandard( uniforms, material );

    }

    }

    // ...

    function refreshUniformsCloth( uniforms, material, environment ) {

    refreshUniformsStandard( uniforms, material, environment );

    uniforms.reflectivity.value = material.reflectivity; // also part of uniforms common

    if ( material.sheen ) {

        uniforms.sheen.value.copy( material.sheen );

    } else {

        uniforms.sheen.value.copy( material.color );
        uniforms.sheen.value.r = Math.sqrt( uniforms.sheen.value.r );
        uniforms.sheen.value.g = Math.sqrt( uniforms.sheen.value.g );
        uniforms.sheen.value.b = Math.sqrt( uniforms.sheen.value.b );

    }

    if ( material.subsurface ) uniforms.subsurface.value.copy( material.subsurface );

    if ( material.brdfCloth ) uniforms.brdfCloth.value = material.brdfCloth;

}
        
    \end{lstlisting}

    De esta forma, tenemos un material con la interfaz nativa de ThreeJs, que tiene acceso a los
    chunks definidos en la clase ShaderChunk y cuyos uniforms y composici\'on de chunks definiremos
    en ShaderLib.\newline

    El BRDF para el componente especular de la iluminaci\'on directa del material Cloth de Filament,
    se utiliza en ThreeJs para a\~nadir opcionalmente un l\'obulo de Sheen al material. Por otra
    parte, el difuso, utiliza en Filament un t\'ermino opcional para ofrecer una aproximaci\'on
    barata el subsurface scattering para iluminaci\'on directa que utiliza Filament.\newline
    
    \begin{lstlisting}[caption=Implementaci\'on del BRDF de iluminaci\'on directa de Filament]
vec3 surfaceShading(const PixelParams pixel, const Light light) {
    vec3 h = normalize(shading_view + light.l);
    float NoL = light.NoL;
    float NoH = saturate(dot(shading_normal, h));
    float LoH = saturate(dot(light.l, h));

    // specular BRDF
    float D = D_Charlie(pixel.roughness, NoH);
    float V = V_Neubelt(shading_NoV, NoL);
    vec3  F = pixel.f0; // f0 is sheen color for Cloth materials 
    vec3 Fr = (D * V) * F;

    // diffuse BRDF
    float diffuse = diffuse(pixel.roughness, shading_NoV, NoL, LoH);
    #if defined(MATERIAL_HAS_SUBSURFACE_COLOR)
    diffuse *= Fd_Wrap(dot(shading_normal, light.l), 0.5);
    #endif

    vec3 Fd = diffuse * pixel.diffuseColor;

#if defined(MATERIAL_HAS_SUBSURFACE_COLOR)
    Fd *= saturate(pixel.subsurfaceColor + NoL);
    vec3 color = ( Fd + Fr * NoL ) * light.colorIntensity
#else
    vec3 color = Fd + Fr * NoL * light.colorIntensity;
#endif

    return color;
}
    \end{lstlisting}
    
    Primero definimos la interfaz de nuestro MeshClothMaterial y pondremos en comparaci\'on con
    el MeshPhysicalMaterial, para analizar los par\'ametros en com\'un y los cambios en sus BRDF
    y ecuaciones de render.\newline
    
    \centering
    \begin{tabular}{| c | c |}
    \hline
    MeshPhysicalMaterial & MeshClothMaterial \\ \hline
    color & color \\
    roughness & roughness \\
    clearCoat & NO \\
    clearCoatRoughness & NO \\
    sheen  & sheen  \\
    NO  & subsurface  \\
    metalness & NO \\
    emissive & emissive \\
    alpha & alpha \\
    lightmap & lightmap \\
    normalMap & normalMap \\ \hline
    \end{tabular}\\
    \egroup